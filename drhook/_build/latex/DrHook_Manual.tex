%% Generated by Sphinx.
\def\sphinxdocclass{report}
\documentclass[letterpaper,10pt,english]{sphinxmanual}
\ifdefined\pdfpxdimen
   \let\sphinxpxdimen\pdfpxdimen\else\newdimen\sphinxpxdimen
\fi \sphinxpxdimen=.75bp\relax
\ifdefined\pdfimageresolution
    \pdfimageresolution= \numexpr \dimexpr1in\relax/\sphinxpxdimen\relax
\fi
%% let collapsible pdf bookmarks panel have high depth per default
\PassOptionsToPackage{bookmarksdepth=5}{hyperref}

\PassOptionsToPackage{booktabs}{sphinx}
\PassOptionsToPackage{colorrows}{sphinx}

\PassOptionsToPackage{warn}{textcomp}
\usepackage[utf8]{inputenc}
\ifdefined\DeclareUnicodeCharacter
% support both utf8 and utf8x syntaxes
  \ifdefined\DeclareUnicodeCharacterAsOptional
    \def\sphinxDUC#1{\DeclareUnicodeCharacter{"#1}}
  \else
    \let\sphinxDUC\DeclareUnicodeCharacter
  \fi
  \sphinxDUC{00A0}{\nobreakspace}
  \sphinxDUC{2500}{\sphinxunichar{2500}}
  \sphinxDUC{2502}{\sphinxunichar{2502}}
  \sphinxDUC{2514}{\sphinxunichar{2514}}
  \sphinxDUC{251C}{\sphinxunichar{251C}}
  \sphinxDUC{2572}{\textbackslash}
\fi
\usepackage{cmap}
\usepackage[T1]{fontenc}
\usepackage{amsmath,amssymb,amstext}
\usepackage{babel}



\usepackage{tgtermes}
\usepackage{tgheros}
\renewcommand{\ttdefault}{txtt}



\usepackage[Bjarne]{fncychap}
\usepackage{sphinx}

\fvset{fontsize=auto}
\usepackage{geometry}


% Include hyperref last.
\usepackage{hyperref}
% Fix anchor placement for figures with captions.
\usepackage{hypcap}% it must be loaded after hyperref.
% Set up styles of URL: it should be placed after hyperref.
\urlstyle{same}


\usepackage{sphinxmessages}
\setcounter{tocdepth}{1}



\title{DrHook Manual}
\date{Mar 05, 2025}
\release{}
\author{ECMWF}
\newcommand{\sphinxlogo}{\vbox{}}
\renewcommand{\releasename}{}
\makeindex
\begin{document}

\ifdefined\shorthandoff
  \ifnum\catcode`\=\string=\active\shorthandoff{=}\fi
  \ifnum\catcode`\"=\active\shorthandoff{"}\fi
\fi

\pagestyle{empty}
\sphinxmaketitle
\pagestyle{plain}
\sphinxtableofcontents
\pagestyle{normal}
\phantomsection\label{\detokenize{index::doc}}


\sphinxstepscope


\chapter{Flags \& Environment variables}
\label{\detokenize{flag/flag:flags-environment-variables}}\label{\detokenize{flag/flag::doc}}

\section{DR\_HOOK\_SHOW\_LOCK}
\label{\detokenize{flag/flag:dr-hook-show-lock}}\label{\detokenize{flag/flag:id1}}

\subsection{Valid Values}
\label{\detokenize{flag/flag:valid-values}}
\begin{DUlineblock}{0em}
\item[] valid\_value ::= ‘0’ | ‘1’
\end{DUlineblock}


\subsection{Purpose}
\label{\detokenize{flag/flag:purpose}}
\sphinxAtStartPar
Used to specify if lock debug info should be output. This is on initialisation only.


\subsection{Notes}
\label{\detokenize{flag/flag:notes}}
\sphinxAtStartPar
This is done by enabling \sphinxcode{\sphinxupquote{OML\_DEBUG}} in the \sphinxstyleemphasis{OML} library, initialising a new \sphinxcode{\sphinxupquote{lockid}}, and then restoring the original value.

\begin{DUlineblock}{0em}
\item[] This debug info takes the following format:
\item[] \sphinxcode{\sphinxupquote{oml\_init\_lockid\_with\_name "\sphinxstyleemphasis{lock\_name}" : \sphinxstyleemphasis{lock\_ID}, \sphinxstyleemphasis{lock\_address}}}
\end{DUlineblock}


\section{OMP\_STACKSIZE}
\label{\detokenize{flag/flag:omp-stacksize}}\label{\detokenize{flag/flag:id3}}

\subsection{Valid Values}
\label{\detokenize{flag/flag:id4}}
\begin{DUlineblock}{0em}
\item[] valid\_value ::= \textless{}number\textgreater{} \textless{}suffix\textgreater{}
\item[] number ::= \textless{}digit\textgreater{} | \textless{}number\textgreater{} \textless{}digit\textgreater{}
\item[] digit ::= ‘0’ | ‘1’ | ‘2’ | ‘3’ | ‘4’ | ‘5’ | ‘6’ | ‘7’ | ‘8’ | ‘9’
\item[] suffix ::= ‘’ | ‘G’ | ‘M’ | ‘K’
\end{DUlineblock}


\subsection{Purpose}
\label{\detokenize{flag/flag:id5}}
\sphinxAtStartPar
Specifies an indicative stack size the master thread should not exceed.


\subsection{Notes}
\label{\detokenize{flag/flag:id6}}
\sphinxAtStartPar
This is an overloaded environment variable, used primarily by \sphinxstyleemphasis{OpenMP}.

\sphinxAtStartPar
It will only be read if {\hyperref[\detokenize{flag/flag:dr-hook-trace-stack}]{\sphinxcrossref{\DUrole{std,std-ref}{DR\_HOOK\_TRACE\_STACK}}}} is also set.

\sphinxAtStartPar
\sphinxcode{\sphinxupquote{OMP\_STACKSIZE}} is scaled by \sphinxcode{\sphinxupquote{opt\_trace\_stack}}, obtained from {\hyperref[\detokenize{flag/flag:dr-hook-trace-stack}]{\sphinxcrossref{\DUrole{std,std-ref}{DR\_HOOK\_TRACE\_STACK}}}}, to give \sphinxcode{\sphinxupquote{drhook\_stacksize\_threshold}}. If \sphinxcode{\sphinxupquote{drhook\_stacksize\_threshold}} is exceeded by the master thread during a \sphinxcode{\sphinxupquote{random\_memstat}} check, an abort will occur.

\sphinxAtStartPar
The stack size can be specified in GiB, MiB, or KiB using the suffix \sphinxcode{\sphinxupquote{G}}, \sphinxcode{\sphinxupquote{M}}, or \sphinxcode{\sphinxupquote{K}} respectively. A lack of suffix will imply the stack size is in bytes. If multiple suffixes are specified, then the largest specified will be chosen. The stack size is truncated at the first occurring suffix.

\sphinxAtStartPar
The size is limited to the size of \sphinxcode{\sphinxupquote{%
\PYG{k+kt}{long}\PYG{+w}{ }\PYG{k+kt}{long}\PYG{+w}{ }\PYG{k+kt}{int}%
}}.

\sphinxAtStartPar
An invalid or negative input size will result in a default of \sphinxcode{\sphinxupquote{0}}.


\section{DR\_HOOK\_CATCH\_SIGNALS}
\label{\detokenize{flag/flag:dr-hook-catch-signals}}\label{\detokenize{flag/flag:id8}}

\subsection{Valid Values}
\label{\detokenize{flag/flag:id9}}
\begin{DUlineblock}{0em}
\item[] valid\_value ::= \textless{}signal\textgreater{} |  \textless{}valid\_value\textgreater{} \textless{}delim\textgreater{} \textless{}signal\textgreater{}
\item[] delim ::= ‘,’ | ‘ ‘ | ‘t’ | ‘/’
\item[] signal ::= ‘\sphinxhyphen{}1’ | \textless{}number\textgreater{}
\item[] number ::= \textless{}digit\textgreater{} | \textless{}number\textgreater{} \textless{}digit\textgreater{} | \textless{}number\textgreater{} ‘0’
\item[] digit ::= ‘1’ | ‘2’ | ‘3’ | ‘4’ | ‘5’ | ‘6’ | ‘7’ | ‘8’ | ‘9’
\end{DUlineblock}


\subsection{Purpose}
\label{\detokenize{flag/flag:id10}}
\sphinxAtStartPar
Specifies a list of signals to be caught and handled by drhook.


\subsection{Notes}
\label{\detokenize{flag/flag:id11}}
\sphinxAtStartPar
If \(1 \leq\) \sphinxcode{\sphinxupquote{signal}} \(\leq\) \sphinxcode{\sphinxupquote{NSIG}}, then it is registered to be caught and handled by drhook \sphinxhyphen{} unless it has been set to ignored by
{\hyperref[\detokenize{flag/flag:dr-hook-ignore-signals}]{\sphinxcrossref{\DUrole{std,std-ref}{DR\_HOOK\_IGNORE\_SIGNALS}}}}. \sphinxcode{\sphinxupquote{NSIG}} is defined in \sphinxcode{\sphinxupquote{signal.h}} and is system dependant.

\sphinxAtStartPar
If \sphinxcode{\sphinxupquote{signal}} is set to \sphinxcode{\sphinxupquote{\sphinxhyphen{}1}}, then all available catchable signals are registered to be caught and handled by drhook. Any value after \sphinxcode{\sphinxupquote{\sphinxhyphen{}1}} will be discarded.

\sphinxAtStartPar
All other values will be silently discarded.


\section{DR\_HOOK\_RESTORE\_DEFAULT\_SIGNALS}
\label{\detokenize{flag/flag:dr-hook-restore-default-signals}}\label{\detokenize{flag/flag:id13}}

\subsection{Valid Values}
\label{\detokenize{flag/flag:id14}}
\begin{DUlineblock}{0em}
\item[] valid\_value ::= \textless{}signal\textgreater{} |  \textless{}valid\_value\textgreater{} \textless{}delim\textgreater{} \textless{}signal\textgreater{}
\item[] delim ::= ‘,’ | ‘ ‘ | ‘t’ | ‘/’
\item[] signal ::= ‘\sphinxhyphen{}1’ | \textless{}number\textgreater{}
\item[] number ::= \textless{}digit\textgreater{} | \textless{}number\textgreater{} \textless{}digit\textgreater{} | \textless{}number\textgreater{} ‘0’
\item[] digit ::= ‘1’ | ‘2’ | ‘3’ | ‘4’ | ‘5’ | ‘6’ | ‘7’ | ‘8’ | ‘9’
\end{DUlineblock}


\subsection{Purpose}
\label{\detokenize{flag/flag:id15}}
\sphinxAtStartPar
Specifies a list of signals to have their default handler restored, if they haven’t been set to be ignored.


\subsection{Notes}
\label{\detokenize{flag/flag:id16}}
\sphinxAtStartPar
If \(1 \leq\) \sphinxcode{\sphinxupquote{signal}} \(\leq\) \sphinxcode{\sphinxupquote{NSIG}}, then it’s default signal handler is restored \sphinxhyphen{} unless it has been set to ignored by
{\hyperref[\detokenize{flag/flag:dr-hook-ignore-signals}]{\sphinxcrossref{\DUrole{std,std-ref}{DR\_HOOK\_IGNORE\_SIGNALS}}}}. \sphinxcode{\sphinxupquote{NSIG}} is defined in \sphinxcode{\sphinxupquote{signal.h}} and is system dependant.

\sphinxAtStartPar
If \sphinxcode{\sphinxupquote{signal}} is set to \sphinxcode{\sphinxupquote{\sphinxhyphen{}1}}, then all available catchable signals have their default signal handler restored. Any value after \sphinxcode{\sphinxupquote{\sphinxhyphen{}1}} will be discarded.

\sphinxAtStartPar
All other values will be silently discarded.


\section{DR\_HOOK\_IGNORE\_SIGNALS}
\label{\detokenize{flag/flag:dr-hook-ignore-signals}}\label{\detokenize{flag/flag:id18}}

\subsection{Valid Values}
\label{\detokenize{flag/flag:id19}}
\begin{DUlineblock}{0em}
\item[] valid\_value ::= \textless{}signal\textgreater{} |  \textless{}valid\_value\textgreater{} \textless{}delim\textgreater{} \textless{}signal\textgreater{}
\item[] delim ::= ‘,’ | ‘ ‘ | ‘t’ | ‘/’
\item[] signal ::= ‘\sphinxhyphen{}1’ | \textless{}number\textgreater{}
\item[] number ::= \textless{}digit\textgreater{} | \textless{}number\textgreater{} \textless{}digit\textgreater{} | \textless{}number\textgreater{} ‘0’
\item[] digit ::= ‘1’ | ‘2’ | ‘3’ | ‘4’ | ‘5’ | ‘6’ | ‘7’ | ‘8’ | ‘9’
\end{DUlineblock}


\subsection{Purpose}
\label{\detokenize{flag/flag:id20}}
\sphinxAtStartPar
Specifies a list of signals to be ignored by drhook. This means drhook will not handle these signals when they occur.


\subsection{Notes}
\label{\detokenize{flag/flag:id21}}
\sphinxAtStartPar
If \(1 \leq\) \sphinxcode{\sphinxupquote{signal}} \(\leq\) \sphinxcode{\sphinxupquote{NSIG}}, then it’s set to inactive and ignored by drhook. \sphinxcode{\sphinxupquote{NSIG}} is defined in \sphinxcode{\sphinxupquote{signal.h}} and is system dependant.

\sphinxAtStartPar
If \sphinxcode{\sphinxupquote{signal}} is set to \sphinxcode{\sphinxupquote{\sphinxhyphen{}1}}, then all available catchable signals are set to inactive and be ignored by drhook. Any value after \sphinxcode{\sphinxupquote{\sphinxhyphen{}1}} will be discarded.

\sphinxAtStartPar
All other values will be silently discarded.


\section{DR\_HOOK\_SILENT}
\label{\detokenize{flag/flag:dr-hook-silent}}\label{\detokenize{flag/flag:id23}}

\subsection{Valid Values}
\label{\detokenize{flag/flag:id24}}
\begin{DUlineblock}{0em}
\item[] valid\_value ::= ‘0’ | ‘1’
\end{DUlineblock}


\subsection{Purpose}
\label{\detokenize{flag/flag:id25}}
\sphinxAtStartPar
Used to enable or disable debug prints to \sphinxcode{\sphinxupquote{STDERR}}.


\subsection{Notes}
\label{\detokenize{flag/flag:id26}}
\sphinxAtStartPar
This is not recommended during production run due to the excessive slowdown caused by printing.


\section{DR\_HOOK\_INIT\_SIGNALS}
\label{\detokenize{flag/flag:dr-hook-init-signals}}\label{\detokenize{flag/flag:id28}}

\subsection{Valid Values}
\label{\detokenize{flag/flag:id29}}
\begin{DUlineblock}{0em}
\item[] valid\_value ::= ‘0’ | ‘1’
\end{DUlineblock}


\subsection{Purpose}
\label{\detokenize{flag/flag:id30}}
\sphinxAtStartPar
Specifies if signals should be initialised via drhook (dangerous, but sometimes necessary).


\subsection{Notes}
\label{\detokenize{flag/flag:id31}}

\section{DR\_HOOK\_SHOW\_PROCESS\_OPTIONS}
\label{\detokenize{flag/flag:dr-hook-show-process-options}}\label{\detokenize{flag/flag:id34}}

\subsection{Valid Values}
\label{\detokenize{flag/flag:id35}}
\begin{DUlineblock}{0em}
\item[] valid\_value ::= \textless{}number\textgreater{} | ‘\sphinxhyphen{}1’
\item[] number ::= \textless{}digit\textgreater{} | \textless{}number\textgreater{} \textless{}digit\textgreater{} | \textless{}number\textgreater{} ‘0’
\item[] digit ::= ‘1’ | ‘2’ | ‘3’ | ‘4’ | ‘5’ | ‘6’ | ‘7’ | ‘8’ | ‘9’
\end{DUlineblock}


\subsection{Purpose}
\label{\detokenize{flag/flag:id36}}
\sphinxAtStartPar
Specifies processIDs for which all options will be output todo\{Where?\}.


\subsection{Notes}
\label{\detokenize{flag/flag:id37}}
\sphinxAtStartPar
If this option isn’t specified, and {\hyperref[\detokenize{flag/flag:dr-hook-silent}]{\sphinxcrossref{\DUrole{std,std-ref}{DR\_HOOK\_SILENT}}}} doesn’t evaluate to \sphinxcode{\sphinxupquote{True}}, then this will default to the process with ID \sphinxcode{\sphinxupquote{1}}.

\sphinxAtStartPar
Specifying \sphinxcode{\sphinxupquote{\sphinxhyphen{}1}} will enable this for all processes.


\section{ATP\_ENABLED}
\label{\detokenize{flag/flag:atp-enabled}}\label{\detokenize{flag/flag:id39}}

\subsection{Valid Values}
\label{\detokenize{flag/flag:id40}}
\begin{DUlineblock}{0em}
\item[] valid\_value ::= ‘0’ | ‘1’
\end{DUlineblock}


\subsection{Purpose}
\label{\detokenize{flag/flag:id41}}
\sphinxAtStartPar
Enable \sphinxhref{https://cpe.ext.hpe.com/docs/debugging-tools/atp.1.html}{Abnormal Termination Processing (ATP)} mode for \sphinxstyleemphasis{Cray} systems. This will pass certain signals to \sphinxstyleabbreviation{ATP} (Abnormal Termination Processing) instead of drhook.


\subsection{Notes}
\label{\detokenize{flag/flag:id42}}
\sphinxAtStartPar
If nothing is specified, then this will default to \sphinxcode{\sphinxupquote{0}}.

\sphinxAtStartPar
This will also cause drhook to check the following flags:
\begin{itemize}
\item {} 
\sphinxAtStartPar
{\hyperref[\detokenize{flag/flag:atp-max-cores}]{\sphinxcrossref{\DUrole{std,std-ref}{ATP\_MAX\_CORES}}}}

\item {} 
\sphinxAtStartPar
{\hyperref[\detokenize{flag/flag:atp-max-analysis-time}]{\sphinxcrossref{\DUrole{std,std-ref}{ATP\_MAX\_ANALYSIS\_TIME}}}}

\item {} 
\sphinxAtStartPar
{\hyperref[\detokenize{flag/flag:atp-ignore-sigterm}]{\sphinxcrossref{\DUrole{std,std-ref}{ATP\_IGNORE\_SIGTERM}}}}

\end{itemize}

\sphinxAtStartPar
The signals passed to \sphinxstyleabbreviation{ATP} are:
\begin{itemize}
\item {} 
\sphinxAtStartPar
\sphinxcode{\sphinxupquote{SIGINT}}

\item {} 
\sphinxAtStartPar
\sphinxcode{\sphinxupquote{SIGFPE}}

\item {} 
\sphinxAtStartPar
\sphinxcode{\sphinxupquote{SIGILL}}

\item {} 
\sphinxAtStartPar
\sphinxcode{\sphinxupquote{SIGTRAP}}

\item {} 
\sphinxAtStartPar
\sphinxcode{\sphinxupquote{SIGABRT}}

\item {} 
\sphinxAtStartPar
\sphinxcode{\sphinxupquote{SIGBUS}}

\item {} 
\sphinxAtStartPar
\sphinxcode{\sphinxupquote{SIGSEGV}}

\item {} 
\sphinxAtStartPar
\sphinxcode{\sphinxupquote{SIGSYS}}

\item {} 
\sphinxAtStartPar
\sphinxcode{\sphinxupquote{SIGXCPU}}

\item {} 
\sphinxAtStartPar
\sphinxcode{\sphinxupquote{SIGXFSZ}}

\item {} 
\sphinxAtStartPar
\sphinxcode{\sphinxupquote{SIGTERM}}
\begin{itemize}
\item {} 
\sphinxAtStartPar
\sphinxcode{\sphinxupquote{SIGTERM}} is only passed to \sphinxstyleabbreviation{ATP} if {\hyperref[\detokenize{flag/flag:atp-ignore-sigterm}]{\sphinxcrossref{\DUrole{std,std-ref}{ATP\_IGNORE\_SIGTERM}}}} is set

\end{itemize}

\end{itemize}


\section{ATP\_MAX\_CORES}
\label{\detokenize{flag/flag:atp-max-cores}}\label{\detokenize{flag/flag:id44}}

\subsection{Valid Values}
\label{\detokenize{flag/flag:id45}}
\begin{DUlineblock}{0em}
\item[] valid\_value ::= \textless{}number\textgreater{}
\item[] number ::= \textless{}digit\textgreater{} | \textless{}number\textgreater{} \textless{}digit\textgreater{}
\item[] digit ::= ‘0’ | ‘1’ | ‘2’ | ‘3’ | ‘4’ | ‘5’ | ‘6’ | ‘7’ | ‘8’ | ‘9’
\end{DUlineblock}


\subsection{Purpose}
\label{\detokenize{flag/flag:id46}}
\sphinxAtStartPar
Used as a multiplier for calculating the spin wait time for all cores to dump if \sphinxstyleabbreviation{ATP} is enabled and handling signals.

\sphinxAtStartPar
Also used along with {\hyperref[\detokenize{flag/flag:atp-enabled}]{\sphinxcrossref{\DUrole{std,std-ref}{ATP\_ENABLED}}}} to determine if any cores are being dumped by \sphinxstyleabbreviation{ATP}.


\subsection{Notes}
\label{\detokenize{flag/flag:id47}}
\sphinxAtStartPar
This is an overloaded environment variable, used primarily for \sphinxstyleabbreviation{ATP}. If nothing is specified, then this will default to \sphinxcode{\sphinxupquote{20}}.

\sphinxAtStartPar
This will only be enable if the {\hyperref[\detokenize{flag/flag:atp-enabled}]{\sphinxcrossref{\DUrole{std,std-ref}{ATP\_ENABLED}}}} flag evaluates to \sphinxcode{\sphinxupquote{True}}.

\sphinxAtStartPar
The size is limited to the size of \sphinxcode{\sphinxupquote{%
\PYG{k+kt}{int}%
}}.


\section{ATP\_MAX\_ANALYSIS\_TIME}
\label{\detokenize{flag/flag:atp-max-analysis-time}}\label{\detokenize{flag/flag:id49}}

\subsection{Valid Values}
\label{\detokenize{flag/flag:id50}}
\begin{DUlineblock}{0em}
\item[] valid\_value ::= \textless{}number\textgreater{}
\item[] number ::= \textless{}digit\textgreater{} | \textless{}number\textgreater{} \textless{}digit\textgreater{}
\item[] digit ::= ‘0’ | ‘1’ | ‘2’ | ‘3’ | ‘4’ | ‘5’ | ‘6’ | ‘7’ | ‘8’ | ‘9’
\end{DUlineblock}


\subsection{Purpose}
\label{\detokenize{flag/flag:id51}}
\sphinxAtStartPar
Used as a base value for calculating the spin wait time for all cores to dump if \sphinxstyleabbreviation{ATP} is enabled and handling signals.


\subsection{Notes}
\label{\detokenize{flag/flag:id52}}
\sphinxAtStartPar
This is an overloaded environment variable, used primarily for \sphinxstyleabbreviation{ATP}. If nothing is specified, then this will default to \sphinxcode{\sphinxupquote{300}}.

\sphinxAtStartPar
This will only be enable if the {\hyperref[\detokenize{flag/flag:atp-enabled}]{\sphinxcrossref{\DUrole{std,std-ref}{ATP\_ENABLED}}}} flag evaluates to \sphinxcode{\sphinxupquote{True}}.

\sphinxAtStartPar
The minimum between \sphinxcode{\sphinxupquote{ATP\_MAX\_ANALYSIS\_TIME}} and \sphinxcode{\sphinxupquote{drhook\_harakiri\_timeout}} (set by {\hyperref[\detokenize{flag/flag:dr-hook-harakiri-timeout}]{\sphinxcrossref{\DUrole{std,std-ref}{DR\_HOOK\_HARAKIRI\_TIMEOUT}}}}) is chosen as the base for calculating the time to spin wait for \sphinxstyleabbreviation{ATP} to finish handling a signal, through the following:

\begin{sphinxVerbatim}[commandchars=\\\{\}]
\PYG{k+kt}{int}\PYG{+w}{ }\PYG{n}{secs}\PYG{+w}{ }\PYG{o}{=}\PYG{+w}{ }\PYG{n}{MIN}\PYG{p}{(}\PYG{n}{drhook\PYGZus{}harakiri\PYGZus{}timeout}\PYG{p}{,}\PYG{+w}{ }\PYG{n}{atp\PYGZus{}max\PYGZus{}analysis\PYGZus{}time}\PYG{p}{)}\PYG{p}{;}
\PYG{n}{secs}\PYG{+w}{ }\PYG{o}{=}\PYG{+w}{ }\PYG{l+m+mi}{60}\PYG{+w}{ }\PYG{o}{+}\PYG{+w}{ }\PYG{n}{MIN}\PYG{p}{(}\PYG{n}{tdiff}\PYG{+w}{ }\PYG{o}{*}\PYG{+w}{ }\PYG{p}{(}\PYG{n}{atp\PYGZus{}max\PYGZus{}cores}\PYG{+w}{ }\PYG{o}{\PYGZhy{}}\PYG{+w}{ }\PYG{l+m+mi}{1}\PYG{p}{)}\PYG{p}{,}\PYG{+w}{ }\PYG{n}{secs}\PYG{p}{)}\PYG{p}{;}
\end{sphinxVerbatim}

\sphinxAtStartPar
The size is limited to the size of \sphinxcode{\sphinxupquote{%
\PYG{k+kt}{int}%
}}.


\section{ATP\_IGNORE\_SIGTERM}
\label{\detokenize{flag/flag:atp-ignore-sigterm}}\label{\detokenize{flag/flag:id54}}

\subsection{Valid Values}
\label{\detokenize{flag/flag:id55}}
\begin{DUlineblock}{0em}
\item[] valid\_value ::= ‘0’ | ‘1’
\end{DUlineblock}


\subsection{Purpose}
\label{\detokenize{flag/flag:id56}}
\sphinxAtStartPar
Determines if \sphinxcode{\sphinxupquote{SIGTERM}} should be handled by drhook or \sphinxstyleabbreviation{ATP}.


\subsection{Notes}
\label{\detokenize{flag/flag:id57}}
\sphinxAtStartPar
This is an overloaded environment variable, used primarily for \sphinxstyleabbreviation{ATP}. If nothing is specified, then this will default to \sphinxcode{\sphinxupquote{0}}.

\sphinxAtStartPar
This will only be enable if the {\hyperref[\detokenize{flag/flag:atp-enabled}]{\sphinxcrossref{\DUrole{std,std-ref}{ATP\_ENABLED}}}} flag evaluates to \sphinxcode{\sphinxupquote{True}}.


\section{DR\_HOOK\_ALLOW\_COREDUMP}
\label{\detokenize{flag/flag:dr-hook-allow-coredump}}\label{\detokenize{flag/flag:id59}}

\subsection{Valid Values}
\label{\detokenize{flag/flag:id60}}
\begin{DUlineblock}{0em}
\item[] valid\_value ::= \textless{}number\textgreater{} | ‘\sphinxhyphen{}1’
\item[] number ::= \textless{}digit\textgreater{} | \textless{}number\textgreater{} \textless{}digit\textgreater{}
\item[] digit ::= ‘0’ | ‘1’ | ‘2’ | ‘3’ | ‘4’ | ‘5’ | ‘6’ | ‘7’ | ‘8’ | ‘9’
\end{DUlineblock}


\subsection{Purpose}
\label{\detokenize{flag/flag:id61}}
\sphinxAtStartPar
Specifies both if core dumping should be enabled, and if so, which processes.


\subsection{Notes}
\label{\detokenize{flag/flag:id62}}
\sphinxAtStartPar
If \sphinxcode{\sphinxupquote{DR\_HOOK\_ALLOW\_COREDUMP}} evaluates to \(1\leq\) \sphinxcode{\sphinxupquote{process}} \(\leq\) \sphinxcode{\sphinxupquote{\# of processes}} then only the specified process has core dumping enabled.

\sphinxAtStartPar
If \sphinxcode{\sphinxupquote{DR\_HOOK\_ALLOW\_COREDUMP}} evaluates to \sphinxcode{\sphinxupquote{\sphinxhyphen{}1}} then core dumping is enabled for all processes.

\sphinxAtStartPar
If \sphinxcode{\sphinxupquote{DR\_HOOK\_ALLOW\_COREDUMP}} evaluates to \sphinxcode{\sphinxupquote{0}} then core dumping is disabled.


\section{DR\_HOOK\_PROFILE}
\label{\detokenize{flag/flag:dr-hook-profile}}\label{\detokenize{flag/flag:id65}}

\subsection{Valid Values}
\label{\detokenize{flag/flag:id66}}
\begin{DUlineblock}{0em}
\item[] valid\_value ::= \textless{}char\textgreater{} | \textless{}valid\_value\textgreater{} \textless{}char\textgreater{}
\item[] char ::= \textless{}letter\textgreater{} | \textless{}digit\textgreater{} | \textless{}symbol\textgreater{}
\item[] letter ::= \textless{}lower\textgreater{} | \textless{}upper\textgreater{}
\item[] upper ::= ‘A’ | ‘B’ | ‘C’ | ‘D’ | ‘E’ | ‘F’ | ‘G’ | ‘H’ | ‘I’ | ‘J’ | ‘K’ | ‘L’ | ‘M’ | ‘N’ | ‘O’ | ‘P’ | ‘Q’ | ‘R’ | ‘S’ | ‘T’ | ‘U’ | ‘V’ | ‘W’ | ‘X’ | ‘Y’ | ‘Z’
\item[] lower ::= ‘a’ | ‘b’ | ‘c’ | ‘d’ | ‘e’ | ‘f’ | ‘g’ | ‘h’ | ‘i’ | ‘j’ | ‘k’ | ‘l’ | ‘m’ | ‘n’ | ‘o’ | ‘p’ | ‘q’ | ‘r’ | ‘s’ | ‘t’ | ‘u’ | ‘v’ | ‘w’ | ‘x’ | ‘y’ | ‘z’
\item[] digit ::= ‘0’ | ‘1’ | ‘2’ | ‘3’ | ‘4’ | ‘5’ | ‘6’ | ‘7’ | ‘8’ | ‘9’
\item[] symbol ::= ‘ \textquotesingle{} | ‘!’ | ‘”’ | ‘\#’ | ‘\$’ | ‘\%’ | ‘\&’ | ‘’’ | ‘(’ | ‘)’ | ‘*’ | ‘+’ | ‘,’ | ‘\sphinxhyphen{}’ | ‘.’ | ‘:’ | ‘;’ | ‘\textless{}’ | ‘=’ | ‘\textgreater{}’ | ‘?’ | ‘@’ | ‘{[}’ | ‘’ | ‘{]}’ | ‘\textasciicircum{}’ | ‘\_’ | ‘\textasciigrave{}’ | ‘\{’ | ‘|’ | ‘\}’ | ‘\textasciitilde{}’
\end{DUlineblock}


\subsection{Purpose}
\label{\detokenize{flag/flag:id67}}
\sphinxAtStartPar
Specifies where to output profiles.


\subsection{Notes}
\label{\detokenize{flag/flag:id68}}
\sphinxAtStartPar
While \sphinxcode{\sphinxupquote{DR\_HOOK\_PROFILE}} is just to be a valid file name, many of the valid symbols are not recommended for obvious reasons.

\sphinxAtStartPar
If a relative path is used, then it will be relative to the rundir of the binary drhook is linked into.

\sphinxAtStartPar
If \sphinxcode{\sphinxupquote{DR\_HOOK\_PROFILE}} doesn’t contain the character \sphinxcode{\sphinxupquote{\%}}, then \sphinxcode{\sphinxupquote{.\%d}} will be appended.

\sphinxAtStartPar
If {\hyperref[\detokenize{flag/flag:dr-hook-profile-proc}]{\sphinxcrossref{\DUrole{std,std-ref}{DR\_HOOK\_PROFILE\_PROC}}}} is set and valid, but \sphinxcode{\sphinxupquote{DR\_HOOK\_PROFILE}} is not set, then \sphinxcode{\sphinxupquote{DR\_HOOK\_PROFILE}} defaults to \sphinxcode{\sphinxupquote{drhook.prof.\%d}}.

\sphinxAtStartPar
If drhook fails to set the process specific patch string either due to an error or process configuration, then it will default to \sphinxcode{\sphinxupquote{drhook.prof.0}}.

\sphinxAtStartPar
\sphinxcode{\sphinxupquote{DR\_HOOK\_PROFILE}} is also indirectly used in \sphinxcode{\sphinxupquote{%
\PYG{n}{get\PYGZus{}memmon\PYGZus{}out}%
}}. \sphinxcode{\sphinxupquote{\sphinxhyphen{}mem}} is simply appended to the previously described path.

\sphinxAtStartPar
\sphinxcode{\sphinxupquote{DR\_HOOK\_PROFILE}} is also indirectly used in \sphinxcode{\sphinxupquote{%
\PYG{n}{get\PYGZus{}csv\PYGZus{}out}%
}} for PAPI mode, enabled with {\hyperref[\detokenize{flag/flag:dr-hook-opt}]{\sphinxcrossref{\DUrole{std,std-ref}{DR\_HOOK\_OPT}}}}. \sphinxcode{\sphinxupquote{.csv}} is simply appended to the previously described path.

\sphinxAtStartPar
This option is only used in profiling and memory profiling modes.


\section{DR\_HOOK\_PROFILE\_PROC}
\label{\detokenize{flag/flag:dr-hook-profile-proc}}\label{\detokenize{flag/flag:id70}}

\subsection{Valid Values}
\label{\detokenize{flag/flag:id71}}
\begin{DUlineblock}{0em}
\item[] valid\_value ::= \textless{}number\textgreater{} | ‘\sphinxhyphen{}1’
\item[] number ::= \textless{}digit\textgreater{} | \textless{}number\textgreater{} \textless{}digit\textgreater{}
\item[] digit ::= ‘0’ | ‘1’ | ‘2’ | ‘3’ | ‘4’ | ‘5’ | ‘6’ | ‘7’ | ‘8’ | ‘9’
\end{DUlineblock}


\subsection{Purpose}
\label{\detokenize{flag/flag:id72}}
\sphinxAtStartPar
Used to specify which process, or all processes, that should output profiling data.


\subsection{Notes}
\label{\detokenize{flag/flag:id73}}
\sphinxAtStartPar
If \sphinxcode{\sphinxupquote{DR\_HOOK\_PROFILE\_PROC}} is \sphinxcode{\sphinxupquote{\sphinxhyphen{}1}}, then all processes will out profiling data. All other valid values will result in the one specified process outputting its data.

\sphinxAtStartPar
If this option isn’t specified, then it will default to \sphinxcode{\sphinxupquote{\sphinxhyphen{}1}}.

\sphinxAtStartPar
The size is limited to the size of \sphinxcode{\sphinxupquote{%
\PYG{k+kt}{double}%
}}.

\sphinxAtStartPar
This option is only used in profiling and memory profiling modes.


\section{DR\_HOOK\_PROFILE\_LIMIT}
\label{\detokenize{flag/flag:dr-hook-profile-limit}}\label{\detokenize{flag/flag:id75}}

\subsection{Valid Values}
\label{\detokenize{flag/flag:id76}}
\begin{DUlineblock}{0em}
\item[] valid\_value ::= \textless{}number\textgreater{} | \textless{}number\textgreater{} ‘.’ \textless{}number\textgreater{}
\item[] number ::= \textless{}digit\textgreater{} | \textless{}number\textgreater{} \textless{}digit\textgreater{}
\item[] digit ::= ‘0’ | ‘1’ | ‘2’ | ‘3’ | ‘4’ | ‘5’ | ‘6’ | ‘7’ | ‘8’ | ‘9’
\end{DUlineblock}


\subsection{Purpose}
\label{\detokenize{flag/flag:id77}}
\sphinxAtStartPar
Specifies at which percentage of the maximum value data points should be dropped from profiling outputs.


\subsection{Notes}
\label{\detokenize{flag/flag:id78}}
\sphinxAtStartPar
If this option isn’t specified, then it will default to \sphinxcode{\sphinxupquote{\sphinxhyphen{}10}}.

\sphinxAtStartPar
The size is limited to the size of \sphinxcode{\sphinxupquote{%
\PYG{k+kt}{double}%
}}.

\sphinxAtStartPar
This option is only used in profiling and memory profiling modes.


\section{DR\_HOOK\_FUNCENTER}
\label{\detokenize{flag/flag:dr-hook-funcenter}}\label{\detokenize{flag/flag:id80}}

\subsection{Valid Values}
\label{\detokenize{flag/flag:id81}}
\begin{DUlineblock}{0em}
\item[] valid\_value ::= \textless{}digit\textgreater{} | \textless{}valid\_value\textgreater{} \textless{}digit\textgreater{}
\item[] digit ::= ‘0’ | ‘1’ | ‘2’ | ‘3’ | ‘4’ | ‘5’ | ‘6’ | ‘7’ | ‘8’ | ‘9’
\end{DUlineblock}


\subsection{Purpose}
\label{\detokenize{flag/flag:id82}}
\sphinxAtStartPar
Used to specify a thread to print memory and stack information to stdout upon entering a function.


\subsection{Notes}
\label{\detokenize{flag/flag:id83}}
\sphinxAtStartPar
If this option isn’t specified, then it will default to \sphinxcode{\sphinxupquote{0}}.

\sphinxAtStartPar
This will also set \sphinxcode{\sphinxupquote{opt\_gethwm}} and \sphinxcode{\sphinxupquote{opt\_getstk}} to \sphinxcode{\sphinxupquote{1}}, that are usually handled by {\hyperref[\detokenize{flag/flag:dr-hook-opt}]{\sphinxcrossref{\DUrole{std,std-ref}{DR\_HOOK\_OPT}}}}.

\sphinxAtStartPar
To see when a process exits a function, use {\hyperref[\detokenize{flag/flag:dr-hook-funcexit}]{\sphinxcrossref{\DUrole{std,std-ref}{DR\_HOOK\_FUNCEXIT}}}}.


\section{DR\_HOOK\_FUNCEXIT}
\label{\detokenize{flag/flag:dr-hook-funcexit}}\label{\detokenize{flag/flag:id85}}

\subsection{Valid Values}
\label{\detokenize{flag/flag:id86}}
\begin{DUlineblock}{0em}
\item[] valid\_value ::= \textless{}digit\textgreater{} | \textless{}valid\_value\textgreater{} \textless{}digit\textgreater{}
\item[] digit ::= ‘0’ | ‘1’ | ‘2’ | ‘3’ | ‘4’ | ‘5’ | ‘6’ | ‘7’ | ‘8’ | ‘9’
\end{DUlineblock}


\subsection{Purpose}
\label{\detokenize{flag/flag:id87}}
\sphinxAtStartPar
Used to specify a thread to print memory and stack information to stdout upon exiting a function.


\subsection{Notes}
\label{\detokenize{flag/flag:id88}}
\sphinxAtStartPar
If this option isn’t specified, then it will default to \sphinxcode{\sphinxupquote{0}}.

\sphinxAtStartPar
This will also set \sphinxcode{\sphinxupquote{opt\_gethwm}} and \sphinxcode{\sphinxupquote{opt\_getstk}} to \sphinxcode{\sphinxupquote{1}}, that are usually handled by {\hyperref[\detokenize{flag/flag:dr-hook-opt}]{\sphinxcrossref{\DUrole{std,std-ref}{DR\_HOOK\_OPT}}}}.

\sphinxAtStartPar
To see when a process enters a function, use {\hyperref[\detokenize{flag/flag:dr-hook-funcenter}]{\sphinxcrossref{\DUrole{std,std-ref}{DR\_HOOK\_FUNCENTER}}}}.


\section{DR\_HOOK\_TIMELINE}
\label{\detokenize{flag/flag:dr-hook-timeline}}\label{\detokenize{flag/flag:id90}}

\subsection{Valid Values}
\label{\detokenize{flag/flag:id91}}
\begin{DUlineblock}{0em}
\item[] valid\_value ::= \textless{}number\textgreater{} | ‘\sphinxhyphen{}1’
\item[] number ::= \textless{}digit\textgreater{} | \textless{}number\textgreater{} \textless{}digit\textgreater{}
\item[] digit ::= ‘0’ | ‘1’ | ‘2’ | ‘3’ | ‘4’ | ‘5’ | ‘6’ | ‘7’ | ‘8’ | ‘9’
\end{DUlineblock}


\subsection{Purpose}
\label{\detokenize{flag/flag:id92}}
\sphinxAtStartPar
Used to enable timeline mode for either all processes or a specific process.


\subsection{Notes}
\label{\detokenize{flag/flag:id93}}
\sphinxAtStartPar
If \sphinxcode{\sphinxupquote{DR\_HOOK\_TIMELINE}} is \sphinxcode{\sphinxupquote{\sphinxhyphen{}1}}, then all processes will have timeline mode enabled. All other non\sphinxhyphen{}zero valid values will result in the one specified process having timeline mode enabled.

\sphinxAtStartPar
If this option isn’t specified, then it will default to \sphinxcode{\sphinxupquote{0}}.

\sphinxAtStartPar
Setting \sphinxcode{\sphinxupquote{DR\_HOOK\_TIMELINE}} to a non\sphinxhyphen{}zero value also causes drhook to check the following options:
\begin{itemize}
\item {} 
\sphinxAtStartPar
{\hyperref[\detokenize{flag/flag:dr-hook-timeline-thread}]{\sphinxcrossref{\DUrole{std,std-ref}{DR\_HOOK\_TIMELINE\_THREAD}}}}

\item {} 
\sphinxAtStartPar
{\hyperref[\detokenize{flag/flag:dr-hook-timeline-format}]{\sphinxcrossref{\DUrole{std,std-ref}{DR\_HOOK\_TIMELINE\_FORMAT}}}}

\item {} 
\sphinxAtStartPar
{\hyperref[\detokenize{flag/flag:dr-hook-timeline-unitno}]{\sphinxcrossref{\DUrole{std,std-ref}{DR\_HOOK\_TIMELINE\_UNITNO}}}}

\item {} 
\sphinxAtStartPar
{\hyperref[\detokenize{flag/flag:dr-hook-timeline-freq}]{\sphinxcrossref{\DUrole{std,std-ref}{DR\_HOOK\_TIMELINE\_FREQ}}}}

\item {} 
\sphinxAtStartPar
{\hyperref[\detokenize{flag/flag:dr-hook-timeline-mb}]{\sphinxcrossref{\DUrole{std,std-ref}{DR\_HOOK\_TIMELINE\_MB}}}}

\end{itemize}


\section{DR\_HOOK\_TIMELINE\_THREAD}
\label{\detokenize{flag/flag:dr-hook-timeline-thread}}\label{\detokenize{flag/flag:id96}}

\subsection{Valid Values}
\label{\detokenize{flag/flag:id97}}
\begin{DUlineblock}{0em}
\item[] valid\_value ::= \textless{}number\textgreater{}
\item[] number ::= \textless{}digit\textgreater{} | \textless{}number\textgreater{} \textless{}digit\textgreater{}
\item[] digit ::= ‘0’ | ‘1’ | ‘2’ | ‘3’ | ‘4’ | ‘5’ | ‘6’ | ‘7’ | ‘8’ | ‘9’
\end{DUlineblock}


\subsection{Purpose}
\label{\detokenize{flag/flag:id98}}
\sphinxAtStartPar
Used to specify which threads should output timeline info upon entry and exit of a region.


\subsection{Notes}
\label{\detokenize{flag/flag:id99}}
\sphinxAtStartPar
If timeline mode is enabled via {\hyperref[\detokenize{flag/flag:dr-hook-timeline}]{\sphinxcrossref{\DUrole{std,std-ref}{DR\_HOOK\_TIMELINE}}}}, then for all threads in the range \(1 \rightarrow n\) (inclusive) drhook will print timeline information on both entry and exit from a region. drhook will also print the sum of \sphinxstyleemphasis{all threads} for the first thread.

\sphinxAtStartPar
If \sphinxcode{\sphinxupquote{DR\_HOOK\_TIMELINE\_THREAD}} is set to \sphinxcode{\sphinxupquote{0}}, then all n will be set so as to cover all threads and the behaviour is identical to the above.

\sphinxAtStartPar
While not strictly valid, any value less than \sphinxcode{\sphinxupquote{0}} will have the same behaviour as \sphinxcode{\sphinxupquote{0}}.

\sphinxAtStartPar
If this option isn’t specified, then it will default to \sphinxcode{\sphinxupquote{1}}.


\section{DR\_HOOK\_TIMELINE\_FORMAT}
\label{\detokenize{flag/flag:dr-hook-timeline-format}}\label{\detokenize{flag/flag:id102}}

\subsection{Valid Values}
\label{\detokenize{flag/flag:id103}}
\begin{DUlineblock}{0em}
\item[] valid\_value ::= ‘0’ | ‘1’
\end{DUlineblock}


\subsection{Purpose}
\label{\detokenize{flag/flag:id104}}
\sphinxAtStartPar
Used to specify if timeline information should be output in value only or verbose mode.


\subsection{Notes}
\label{\detokenize{flag/flag:id105}}
\sphinxAtStartPar
This option is only available if timeline mode is enabled via {\hyperref[\detokenize{flag/flag:dr-hook-timeline}]{\sphinxcrossref{\DUrole{std,std-ref}{DR\_HOOK\_TIMELINE}}}}.

\sphinxAtStartPar
If \sphinxcode{\sphinxupquote{DR\_HOOK\_TIMELINE\_FORMAT}} is \sphinxcode{\sphinxupquote{1}}, then the following value only formatter will be used:

\sphinxAtStartPar
\sphinxcode{\sphinxupquote{%
\PYG{l+s}{\PYGZdq{}}\PYG{l+s}{\PYGZpc{}.6f \PYGZpc{}.4g \PYGZpc{}.4g \PYGZpc{}.4g \PYGZpc{}.4g}\PYG{l+s}{\PYGZdq{}}%
}}

\sphinxAtStartPar
Otherwise, the verbose formatter is used:

\sphinxAtStartPar
\sphinxcode{\sphinxupquote{%
\PYG{l+s}{\PYGZdq{}}\PYG{l+s}{wall=\PYGZpc{}.6f cpu=\PYGZpc{}.4g hwm=\PYGZpc{}.4g rss=\PYGZpc{}.4g curheap=\PYGZpc{}.4g stack=\PYGZpc{}.4g vmpeak=\PYGZpc{}.4g pag=\PYGZpc{}lld}\PYG{l+s}{\PYGZdq{}}%
}}

\sphinxAtStartPar
If this option isn’t specified, then it will default to \sphinxcode{\sphinxupquote{1}}.


\section{DR\_HOOK\_TIMELINE\_UNITNO}
\label{\detokenize{flag/flag:dr-hook-timeline-unitno}}\label{\detokenize{flag/flag:id107}}

\subsection{Valid Values}
\label{\detokenize{flag/flag:id108}}
\begin{DUlineblock}{0em}
\item[] valid\_value ::= \textless{}digit\textgreater{} | \textless{}valid\_value\textgreater{} \textless{}digit\textgreater{}
\item[] digit ::= ‘0’ | ‘1’ | ‘2’ | ‘3’ | ‘4’ | ‘5’ | ‘6’ | ‘7’ | ‘8’ | ‘9’
\end{DUlineblock}


\subsection{Purpose}
\label{\detokenize{flag/flag:id109}}
\sphinxAtStartPar
Specifies the unit number to be used for timeline output.


\subsection{Notes}
\label{\detokenize{flag/flag:id110}}
\sphinxAtStartPar
This option is only available if timeline mode is enabled via {\hyperref[\detokenize{flag/flag:dr-hook-timeline}]{\sphinxcrossref{\DUrole{std,std-ref}{DR\_HOOK\_TIMELINE}}}}.

\sphinxAtStartPar
\sphinxcode{\sphinxupquote{DR\_HOOK\_TIMELINE\_UNITNO}} must be an integer between \sphinxcode{\sphinxupquote{1}} and \sphinxcode{\sphinxupquote{99}}. Some unit numbers are reserved: \sphinxcode{\sphinxupquote{5}} is standard input, \sphinxcode{\sphinxupquote{6}} is standard output.

\sphinxAtStartPar
As the value of \sphinxcode{\sphinxupquote{DR\_HOOK\_TIMELINE\_UNITNO}} has to be interpreted by \sphinxcode{\sphinxupquote{%
\PYG{n}{atoi}\PYG{p}{(}\PYG{p}{)}%
}}, only integer unit numbers are valid.

\sphinxAtStartPar
If \sphinxcode{\sphinxupquote{0}} is specified, then timeline outputs are silently ignored.

\sphinxAtStartPar
If this option isn’t specified, then it will default to \sphinxcode{\sphinxupquote{6}}.


\section{DR\_HOOK\_TIMELINE\_FREQ}
\label{\detokenize{flag/flag:dr-hook-timeline-freq}}\label{\detokenize{flag/flag:id112}}

\subsection{Valid Values}
\label{\detokenize{flag/flag:id113}}
\begin{DUlineblock}{0em}
\item[] valid\_value ::= \textless{}digit\textgreater{} | \textless{}valid\_value\textgreater{} \textless{}digit\textgreater{} | \textless{}valid\_value\textgreater{} ‘0’
\item[] digit ::= ‘1’ | ‘2’ | ‘3’ | ‘4’ | ‘5’ | ‘6’ | ‘7’ | ‘8’ | ‘9’
\end{DUlineblock}


\subsection{Purpose}
\label{\detokenize{flag/flag:id114}}
\sphinxAtStartPar
Used to specify how many DR\_HOOK\_TIMELINE\_FREQ$^{\text{th}}$ regions entries/exits should be output during timeline mode.


\subsection{Notes}
\label{\detokenize{flag/flag:id115}}
\sphinxAtStartPar
This option is only available if timeline mode is enabled via {\hyperref[\detokenize{flag/flag:dr-hook-timeline}]{\sphinxcrossref{\DUrole{std,std-ref}{DR\_HOOK\_TIMELINE}}}}.

\sphinxAtStartPar
If a drhook region is entered/exited and it is the n$^{\text{th}}$ entry/exit, where n is a multiple of \sphinxcode{\sphinxupquote{DR\_HOOK\_TIMELINE}}, then the timeline information will be output.

\sphinxAtStartPar
If a drhook region is entered/exited and it isn’t the n$^{\text{th}}$ call, but the resident set size varies by more than {\hyperref[\detokenize{flag/flag:dr-hook-timeline-mb}]{\sphinxcrossref{\DUrole{std,std-ref}{DR\_HOOK\_TIMELINE\_MB}}}}, then it is output anyway.

\sphinxAtStartPar
A value less than \sphinxcode{\sphinxupquote{1}} will silently disable timeline mode.

\sphinxAtStartPar
The size is limited to the size of \sphinxcode{\sphinxupquote{%
\PYG{k+kt}{long}\PYG{+w}{ }\PYG{k+kt}{long}\PYG{+w}{ }\PYG{k+kt}{int}%
}}.

\sphinxAtStartPar
If this option isn’t specified, then it will default to \sphinxcode{\sphinxupquote{1000000}}.


\section{DR\_HOOK\_TIMELINE\_MB}
\label{\detokenize{flag/flag:dr-hook-timeline-mb}}\label{\detokenize{flag/flag:id117}}

\subsection{Valid Values}
\label{\detokenize{flag/flag:id118}}
\begin{DUlineblock}{0em}
\item[] valid\_value ::= \textless{}number\textgreater{} | \textless{}number\textgreater{} ‘.’ \textless{}number\textgreater{}
\item[] number ::= \textless{}digit\textgreater{} | \textless{}number\textgreater{} \textless{}digit\textgreater{}
\item[] digit ::= ‘0’ | ‘1’ | ‘2’ | ‘3’ | ‘4’ | ‘5’ | ‘6’ | ‘7’ | ‘8’ | ‘9’
\end{DUlineblock}


\subsection{Purpose}
\label{\detokenize{flag/flag:id119}}
\sphinxAtStartPar
This specifies how much the resident set size needs to vary by, in MB, before {\hyperref[\detokenize{flag/flag:dr-hook-timeline-freq}]{\sphinxcrossref{\DUrole{std,std-ref}{DR\_HOOK\_TIMELINE\_FREQ}}}} is overridden.


\subsection{Notes}
\label{\detokenize{flag/flag:id121}}
\sphinxAtStartPar
This option is only available if timeline mode is enabled via {\hyperref[\detokenize{flag/flag:dr-hook-timeline}]{\sphinxcrossref{\DUrole{std,std-ref}{DR\_HOOK\_TIMELINE}}}}.

\sphinxAtStartPar
A value less than \sphinxcode{\sphinxupquote{0}} will be set to \sphinxcode{\sphinxupquote{1.0}}.

\sphinxAtStartPar
The size is limited to the size of \sphinxcode{\sphinxupquote{%
\PYG{k+kt}{double}%
}}.

\sphinxAtStartPar
If this option isn’t specified, then it will default to \sphinxcode{\sphinxupquote{1.0}}.


\section{DR\_HOOK\_TRACE\_STACK}
\label{\detokenize{flag/flag:dr-hook-trace-stack}}\label{\detokenize{flag/flag:id123}}

\subsection{Valid Values}
\label{\detokenize{flag/flag:id124}}
\begin{DUlineblock}{0em}
\item[] valid\_value ::= \textless{}number\textgreater{} | \textless{}number\textgreater{} ‘.’ \textless{}number\textgreater{}
\item[] number ::= \textless{}digit\textgreater{} | \textless{}number\textgreater{} \textless{}digit\textgreater{}
\item[] digit ::= ‘0’ | ‘1’ | ‘2’ | ‘3’ | ‘4’ | ‘5’ | ‘6’ | ‘7’ | ‘8’ | ‘9’
\end{DUlineblock}


\subsection{Purpose}
\label{\detokenize{flag/flag:id125}}
\sphinxAtStartPar
A multiplier for {\hyperref[\detokenize{flag/flag:omp-stacksize}]{\sphinxcrossref{\DUrole{std,std-ref}{OMP\_STACKSIZE}}}}, to monitor high master thread stack usage.


\subsection{Notes}
\label{\detokenize{flag/flag:id126}}
\sphinxAtStartPar
As described for {\hyperref[\detokenize{flag/flag:omp-stacksize}]{\sphinxcrossref{\DUrole{std,std-ref}{OMP\_STACKSIZE}}}}, \sphinxcode{\sphinxupquote{DR\_HOOK\_TRACE\_STACK}} is used to scale the value of {\hyperref[\detokenize{flag/flag:omp-stacksize}]{\sphinxcrossref{\DUrole{std,std-ref}{OMP\_STACKSIZE}}}} to give \sphinxcode{\sphinxupquote{%
\PYG{n}{drhook\PYGZus{}stacksize\PYGZus{}threshold}%
}}. If \sphinxcode{\sphinxupquote{%
\PYG{n}{drhook\PYGZus{}stacksize\PYGZus{}threshold}%
}} is exceeded by the master thread during a \sphinxcode{\sphinxupquote{random\_memstat}} check, an abort will occur.

\sphinxAtStartPar
\sphinxcode{\sphinxupquote{DR\_HOOK\_TRACE\_STACK}} being defined and non\sphinxhyphen{}zero also implies \sphinxcode{\sphinxupquote{%
\PYG{n}{opt\PYGZus{}random\PYGZus{}memstat}%
}} has a default value of \sphinxcode{\sphinxupquote{1}} (meaning it will always trigger a \sphinxcode{\sphinxupquote{random\_memstat}} check on entry to a drhook region). However, if {\hyperref[\detokenize{flag/flag:dr-hook-random-memstat}]{\sphinxcrossref{\DUrole{std,std-ref}{DR\_HOOK\_RANDOM\_MEMSTAT}}}} is defined, it will override this value.

\sphinxAtStartPar
A value less than \sphinxcode{\sphinxupquote{0}} will be set to \sphinxcode{\sphinxupquote{0}}. Additionally, \sphinxcode{\sphinxupquote{drhook\_stacksize\_threshold}} won’t be scaled, and \sphinxcode{\sphinxupquote{%
\PYG{n}{opt\PYGZus{}random\PYGZus{}memstat}%
}} is not set to \sphinxcode{\sphinxupquote{1}}.

\sphinxAtStartPar
The size is limited to the size of \sphinxcode{\sphinxupquote{%
\PYG{k+kt}{double}%
}}.

\sphinxAtStartPar
If this option isn’t specified, then it will default to \sphinxcode{\sphinxupquote{0}}.


\section{DR\_HOOK\_RANDOM\_MEMSTAT}
\label{\detokenize{flag/flag:dr-hook-random-memstat}}\label{\detokenize{flag/flag:id128}}

\subsection{Valid Values}
\label{\detokenize{flag/flag:id129}}
\begin{DUlineblock}{0em}
\item[] valid\_value ::= \textless{}digit\textgreater{} | \textless{}valid\_value\textgreater{} \textless{}digit\textgreater{}
\item[] digit ::= ‘0’ | ‘1’ | ‘2’ | ‘3’ | ‘4’ | ‘5’ | ‘6’ | ‘7’ | ‘8’ | ‘9’
\end{DUlineblock}


\subsection{Purpose}
\label{\detokenize{flag/flag:id130}}
\sphinxAtStartPar
Used to enable random memstat checks, and how often to do it.


\subsection{Notes}
\label{\detokenize{flag/flag:id131}}
\sphinxAtStartPar
The random memstat check is done when a drhook region is entered and \sphinxcode{\sphinxupquote{%
\PYG{n}{rand}\PYG{p}{(}\PYG{p}{)}\PYG{+w}{ }\PYG{o}{\PYGZpc{}}\PYG{+w}{ }\PYG{n}{opt\PYGZus{}random\PYGZus{}memstat}\PYG{+w}{ }\PYG{o}{=}\PYG{o}{=}\PYG{+w}{ }\PYG{l+m+mi}{0}%
}}, or unconditionally when the feature is explicitly, or implicitly (see {\hyperref[\detokenize{flag/flag:dr-hook-trace-stack}]{\sphinxcrossref{\DUrole{std,std-ref}{DR\_HOOK\_TRACE\_STACK}}}}) enabled.

\sphinxAtStartPar
Due to the implementation of the random check, the value of \sphinxcode{\sphinxupquote{DR\_HOOK\_RANDOM\_MEMSTAT}} is important in ways that may not be immediately obvious. For example, a value of \sphinxcode{\sphinxupquote{2}} will result in a check every other entry to a drhook region on average. Whereas a sufficiently large number will only result in a check every \sphinxcode{\sphinxupquote{1/RAND\_MAX}} times on average.

\sphinxAtStartPar
A value greater than \sphinxcode{\sphinxupquote{%
\PYG{n}{RAND\PYGZus{}MAX}%
}} will be set to \sphinxcode{\sphinxupquote{%
\PYG{n}{RAND\PYGZus{}MAX}%
}}.

\sphinxAtStartPar
A value less than \sphinxcode{\sphinxupquote{0}} will be set to \sphinxcode{\sphinxupquote{0}}, effectively disabling the feature.

\sphinxAtStartPar
The size is limited to the size of \sphinxcode{\sphinxupquote{%
\PYG{k+kt}{int}%
}}.

\sphinxAtStartPar
If this option isn’t specified, then it will default to \sphinxcode{\sphinxupquote{0}}.


\section{DR\_HOOK\_HASHBITS}
\label{\detokenize{flag/flag:dr-hook-hashbits}}\label{\detokenize{flag/flag:id133}}

\subsection{Valid Values}
\label{\detokenize{flag/flag:id134}}
\begin{DUlineblock}{0em}
\item[] valid\_value ::= \textless{}digit\textgreater{} | \textless{}valid\_value\textgreater{} \textless{}digit\textgreater{}
\item[] digit ::= ‘0’ | ‘1’ | ‘2’ | ‘3’ | ‘4’ | ‘5’ | ‘6’ | ‘7’ | ‘8’ | ‘9’
\end{DUlineblock}


\subsection{Purpose}
\label{\detokenize{flag/flag:id135}}

\subsection{Notes}
\label{\detokenize{flag/flag:id137}}
\sphinxAtStartPar
A value greater than \sphinxcode{\sphinxupquote{RAND\_MAX}} will be set to \sphinxcode{\sphinxupquote{RAND\_MAX}}.

\sphinxAtStartPar
A value less than \sphinxcode{\sphinxupquote{0}} will be set to \sphinxcode{\sphinxupquote{0}}.

\begin{DUlineblock}{0em}
\item[] The value of \sphinxcode{\sphinxupquote{DR\_HOOK\_HASHBITS}}, \sphinxcode{\sphinxupquote{%
\PYG{n}{nhash}%
}}, is also used to update the values of \sphinxcode{\sphinxupquote{%
\PYG{n}{hashsize}%
}} and \sphinxcode{\sphinxupquote{%
\PYG{n}{hashmask}%
}}. This is done for \sphinxcode{\sphinxupquote{%
\PYG{n}{hashsize}%
}} via the following:
\item[] \sphinxcode{\sphinxupquote{%
\PYG{k}{static}\PYG{+w}{ }\PYG{k+kt}{unsigned}\PYG{+w}{ }\PYG{k+kt}{int}\PYG{+w}{ }\PYG{n}{hashsize}\PYG{+w}{ }\PYG{o}{=}\PYG{+w}{ }\PYG{p}{(}\PYG{p}{(}\PYG{k+kt}{unsigned}\PYG{+w}{ }\PYG{k+kt}{int}\PYG{p}{)}\PYG{l+m+mi}{1}\PYG{o}{\PYGZlt{}}\PYG{o}{\PYGZlt{}}\PYG{p}{(}\PYG{n}{nhash}\PYG{p}{)}\PYG{p}{)}\PYG{p}{;}%
}}
\item[] For \sphinxcode{\sphinxupquote{hashmask}} it is:
\item[] \sphinxcode{\sphinxupquote{%
\PYG{k}{static}\PYG{+w}{ }\PYG{k+kt}{unsigned}\PYG{+w}{ }\PYG{k+kt}{int}\PYG{+w}{ }\PYG{n}{hashmask}\PYG{+w}{ }\PYG{o}{=}\PYG{+w}{ }\PYG{p}{(}\PYG{p}{(}\PYG{p}{(}\PYG{k+kt}{unsigned}\PYG{+w}{ }\PYG{k+kt}{int}\PYG{p}{)}\PYG{l+m+mi}{1}\PYG{o}{\PYGZlt{}}\PYG{o}{\PYGZlt{}}\PYG{p}{(}\PYG{n}{nhash}\PYG{p}{)}\PYG{p}{)}\PYG{l+m+mi}{\PYGZhy{}1}\PYG{p}{)}\PYG{p}{;}%
}}
\end{DUlineblock}

\sphinxAtStartPar
The size is limited to the size of \sphinxcode{\sphinxupquote{%
\PYG{k+kt}{int}%
}}.

\sphinxAtStartPar
If this option isn’t specified, then it will default to the definition of \sphinxcode{\sphinxupquote{NHASH}}, typically \sphinxcode{\sphinxupquote{16}}. As such, \sphinxcode{\sphinxupquote{hashsize}} and \sphinxcode{\sphinxupquote{hashmask}} default to \sphinxcode{\sphinxupquote{65536}} and \sphinxcode{\sphinxupquote{65535}}, respectively.


\section{DR\_HOOK\_NCALLSTACK}
\label{\detokenize{flag/flag:dr-hook-ncallstack}}\label{\detokenize{flag/flag:id139}}

\subsection{Valid Values}
\label{\detokenize{flag/flag:id140}}
\begin{DUlineblock}{0em}
\item[] valid\_value ::= \textless{}digit\textgreater{} | \textless{}valid\_value\textgreater{} \textless{}digit\textgreater{}
\item[] digit ::= ‘0’ | ‘1’ | ‘2’ | ‘3’ | ‘4’ | ‘5’ | ‘6’ | ‘7’ | ‘8’ | ‘9’
\end{DUlineblock}


\subsection{Purpose}
\label{\detokenize{flag/flag:id141}}

\subsection{Notes}
\label{\detokenize{flag/flag:id143}}
\sphinxAtStartPar
This is an overloaded value with 2 functions. The first is to set the precision used by drhook, which also determines which method is used to X. A performance penalty is incurred when using the single precision mode (valid non\sphinxhyphen{}zero values), and so is not recommended.

\sphinxAtStartPar
The second function of \sphinxcode{\sphinxupquote{DR\_HOOK\_NCALLSTACK}}, is to act as a parameter when in single precision mode. This parameter sets the maximum stack depth allowed. You must be careful when selecting this that a sufficient depth is chosen. If the depth is exceeded, then drhook will abort.

\sphinxAtStartPar
A value less than \sphinxcode{\sphinxupquote{0}} will be set to \sphinxcode{\sphinxupquote{0}}, effectively selecting double precision.

\sphinxAtStartPar
The size is limited to the size of \sphinxcode{\sphinxupquote{%
\PYG{k+kt}{int}%
}}.

\sphinxAtStartPar
If this option isn’t specified, then it will default to \sphinxcode{\sphinxupquote{0}}, which enables double precision mode.


\section{DR\_HOOK\_HARAKIRI\_TIMEOUT}
\label{\detokenize{flag/flag:dr-hook-harakiri-timeout}}\label{\detokenize{flag/flag:id146}}

\subsection{Valid Values}
\label{\detokenize{flag/flag:id147}}
\begin{DUlineblock}{0em}
\item[] valid\_value ::= \textless{}digit\textgreater{} | \textless{}valid\_value\textgreater{} \textless{}digit\textgreater{}
\item[] digit ::= ‘0’ | ‘1’ | ‘2’ | ‘3’ | ‘4’ | ‘5’ | ‘6’ | ‘7’ | ‘8’ | ‘9’
\end{DUlineblock}


\subsection{Purpose}
\label{\detokenize{flag/flag:id148}}
\sphinxAtStartPar
A timeout for when to kill threads that may have hung.


\subsection{Notes}
\label{\detokenize{flag/flag:id149}}
\sphinxAtStartPar
\sphinxcode{\sphinxupquote{DR\_HOOK\_HARAKIRI\_TIMEOUT}} is used in three places. The first is when a signal is first caught by drhook; it is used as the timeout for \sphinxcode{\sphinxupquote{%
\PYG{n}{alarm}\PYG{p}{(}\PYG{p}{)}%
}}. This is to prevent hangs by triggering a second signal when the \sphinxcode{\sphinxupquote{%
\PYG{n}{alarm}\PYG{p}{(}\PYG{p}{)}%
}} expires. This second signal then takes an alternative path for subsequent signals where the second use case of \sphinxcode{\sphinxupquote{DR\_HOOK\_HARAKIRI\_TIMEOUT}} occurs. Here it is used to spin for \sphinxcode{\sphinxupquote{DR\_HOOK\_HARAKIRI\_TIMEOUT}} \(+ 60\) seconds before the thread is killed using \sphinxcode{\sphinxupquote{SIGKILL}}. This is to allow for tracebacks to complete.

\sphinxAtStartPar
The third place it is used, is in conjunction with {\hyperref[\detokenize{flag/flag:atp-max-analysis-time}]{\sphinxcrossref{\DUrole{std,std-ref}{ATP\_MAX\_ANALYSIS\_TIME}}}}. Here it is used in the following way to allow \sphinxstyleabbreviation{ATP} to dump cores:

\begin{sphinxVerbatim}[commandchars=\\\{\}]
\PYG{k+kt}{int}\PYG{+w}{ }\PYG{n}{secs}\PYG{+w}{ }\PYG{o}{=}\PYG{+w}{ }\PYG{n}{MIN}\PYG{p}{(}\PYG{n}{drhook\PYGZus{}harakiri\PYGZus{}timeout}\PYG{p}{,}\PYG{+w}{ }\PYG{n}{atp\PYGZus{}max\PYGZus{}analysis\PYGZus{}time}\PYG{p}{)}\PYG{p}{;}
\PYG{n}{secs}\PYG{+w}{ }\PYG{o}{=}\PYG{+w}{ }\PYG{l+m+mi}{60}\PYG{+w}{ }\PYG{o}{+}\PYG{+w}{ }\PYG{n}{MIN}\PYG{p}{(}\PYG{n}{tdiff}\PYG{+w}{ }\PYG{o}{*}\PYG{+w}{ }\PYG{p}{(}\PYG{n}{atp\PYGZus{}max\PYGZus{}cores}\PYG{+w}{ }\PYG{o}{\PYGZhy{}}\PYG{+w}{ }\PYG{l+m+mi}{1}\PYG{p}{)}\PYG{p}{,}\PYG{+w}{ }\PYG{n}{secs}\PYG{p}{)}\PYG{p}{;}
\end{sphinxVerbatim}

\sphinxAtStartPar
The timeout is specified in seconds.

\sphinxAtStartPar
A value less than \sphinxcode{\sphinxupquote{0}} will be set to the definition of \sphinxcode{\sphinxupquote{drhook\_harakiri\_timeout\_default}}, typically \sphinxcode{\sphinxupquote{500}}.

\sphinxAtStartPar
The size is limited to the size of \sphinxcode{\sphinxupquote{%
\PYG{k+kt}{int}%
}}.

\sphinxAtStartPar
If this option isn’t specified, then it will default to the definition of \sphinxcode{\sphinxupquote{drhook\_harakiri\_timeout\_default}}.


\section{DR\_HOOK\_USE\_LOCKFILE}
\label{\detokenize{flag/flag:dr-hook-use-lockfile}}\label{\detokenize{flag/flag:id151}}

\subsection{Valid Values}
\label{\detokenize{flag/flag:id152}}
\begin{DUlineblock}{0em}
\item[] valid\_value ::= ‘0’ | ‘1’
\end{DUlineblock}


\subsection{Purpose}
\label{\detokenize{flag/flag:id153}}
\sphinxAtStartPar
Toggles the use of a lockfile, \sphinxcode{\sphinxupquote{drhook\_lock}}, for outputting which thread handled a signal first.


\subsection{Notes}
\label{\detokenize{flag/flag:id154}}
\sphinxAtStartPar
When enabled, this will enable a lockfile when handling signals in drhook. This lockfile, \sphinxcode{\sphinxupquote{drhook\_lock}}, will contain the number of the thread which got the lock first.

\sphinxAtStartPar
\sphinxcode{\sphinxupquote{DR\_HOOK\_USE\_LOCKFILE}} having a value evaluating to \sphinxcode{\sphinxupquote{1}}, will also disable some output regarding the use of {\hyperref[\detokenize{flag/flag:dr-hook-use-lockfile}]{\sphinxcrossref{\DUrole{std,std-ref}{DR\_HOOK\_USE\_LOCKFILE}}}} in it’s alternative path for subsequent signals.

\sphinxAtStartPar
Any non\sphinxhyphen{}zero valid integer value will be set to \sphinxcode{\sphinxupquote{1}}.

\sphinxAtStartPar
If this option isn’t specified, then it will default to \sphinxcode{\sphinxupquote{1}}.


\section{DR\_HOOK\_TRAPFPE}
\label{\detokenize{flag/flag:dr-hook-trapfpe}}\label{\detokenize{flag/flag:id158}}

\subsection{Valid Values}
\label{\detokenize{flag/flag:id159}}
\begin{DUlineblock}{0em}
\item[] valid\_value ::= ‘0’ | ‘1’
\end{DUlineblock}


\subsection{Purpose}
\label{\detokenize{flag/flag:id160}}
\sphinxAtStartPar
Toggles if floating point exceptions should be trapped, regardless of compiler settings.


\subsection{Notes}
\label{\detokenize{flag/flag:id161}}
\sphinxAtStartPar
If \sphinxcode{\sphinxupquote{DR\_HOOK\_TRAPFPE}} evaluates to \sphinxcode{\sphinxupquote{1}}, then drhook will trap floating point exceptions regardless of compilation settings. A value of \sphinxcode{\sphinxupquote{0}} will instead rely on the compiler flags used, e.g. \sphinxcode{\sphinxupquote{\sphinxhyphen{}Ktrap=fp}}.

\sphinxAtStartPar
Any non\sphinxhyphen{}zero valid integer value will be set to \sphinxcode{\sphinxupquote{1}}.

\sphinxAtStartPar
If this option isn’t specified, then it will default to \sphinxcode{\sphinxupquote{1}}.


\section{DR\_HOOK\_TRAPFPE\_INVALID}
\label{\detokenize{flag/flag:dr-hook-trapfpe-invalid}}\label{\detokenize{flag/flag:id164}}

\subsection{Valid Values}
\label{\detokenize{flag/flag:id165}}
\begin{DUlineblock}{0em}
\item[] valid\_value ::= ‘0’ | ‘1’
\end{DUlineblock}


\subsection{Purpose}
\label{\detokenize{flag/flag:id166}}
\sphinxAtStartPar
Toggles whether invalid operation floating point exceptions should be trapped.


\subsection{Notes}
\label{\detokenize{flag/flag:id167}}
\sphinxAtStartPar
This will only be enabled if {\hyperref[\detokenize{flag/flag:dr-hook-trapfpe}]{\sphinxcrossref{\DUrole{std,std-ref}{DR\_HOOK\_TRAPFPE}}}} evaluates to \sphinxcode{\sphinxupquote{1}} or the compiler had trapping of floating point exceptions enabled.

\sphinxAtStartPar
These invalid operations are defined by IEEE 754 standard.

\sphinxAtStartPar
If the exception is not trapped, then the result of the operation is \sphinxcode{\sphinxupquote{NaN}}.

\sphinxAtStartPar
Any non\sphinxhyphen{}zero valid integer value will be set to \sphinxcode{\sphinxupquote{1}}.

\sphinxAtStartPar
If this option isn’t specified, then it will default to \sphinxcode{\sphinxupquote{1}}.


\section{DR\_HOOK\_TRAPFPE\_DIVBYZERO}
\label{\detokenize{flag/flag:dr-hook-trapfpe-divbyzero}}\label{\detokenize{flag/flag:id169}}

\subsection{Valid Values}
\label{\detokenize{flag/flag:id170}}
\begin{DUlineblock}{0em}
\item[] valid\_value ::= ‘0’ | ‘1’
\end{DUlineblock}


\subsection{Purpose}
\label{\detokenize{flag/flag:id171}}
\sphinxAtStartPar
Toggles whether divide by zero floating point exceptions should be trapped.


\subsection{Notes}
\label{\detokenize{flag/flag:id172}}
\sphinxAtStartPar
This will only be enabled if {\hyperref[\detokenize{flag/flag:dr-hook-trapfpe}]{\sphinxcrossref{\DUrole{std,std-ref}{DR\_HOOK\_TRAPFPE}}}} evaluates to \sphinxcode{\sphinxupquote{1}} or the compiler had trapping of floating point exceptions enabled.

\sphinxAtStartPar
This exception occurs when a finite nonzero number is divided by zero.

\sphinxAtStartPar
Any non\sphinxhyphen{}zero valid integer value will be set to \sphinxcode{\sphinxupquote{1}}.

\sphinxAtStartPar
If this option isn’t specified, then it will default to \sphinxcode{\sphinxupquote{1}}.


\section{DR\_HOOK\_TRAPFPE\_OVERFLOW}
\label{\detokenize{flag/flag:dr-hook-trapfpe-overflow}}\label{\detokenize{flag/flag:id174}}

\subsection{Valid Values}
\label{\detokenize{flag/flag:id175}}
\begin{DUlineblock}{0em}
\item[] valid\_value ::= ‘0’ | ‘1’
\end{DUlineblock}


\subsection{Purpose}
\label{\detokenize{flag/flag:id176}}
\sphinxAtStartPar
Toggles whether overflow floating point exceptions should be trapped.


\subsection{Notes}
\label{\detokenize{flag/flag:id177}}
\sphinxAtStartPar
This will only be enabled if {\hyperref[\detokenize{flag/flag:dr-hook-trapfpe}]{\sphinxcrossref{\DUrole{std,std-ref}{DR\_HOOK\_TRAPFPE}}}} evaluates to \sphinxcode{\sphinxupquote{1}} or the compiler had trapping of floating point exceptions enabled.

\sphinxAtStartPar
This exception occurs when the result of a calculation cannot be represented as a finite value in the precision format of the destination. This behaviour is affected by rounding modes.

\sphinxAtStartPar
Any non\sphinxhyphen{}zero valid integer value will be set to \sphinxcode{\sphinxupquote{1}}.

\sphinxAtStartPar
If this option isn’t specified, then it will default to \sphinxcode{\sphinxupquote{1}}.


\section{DR\_HOOK\_TIMED\_KILL}
\label{\detokenize{flag/flag:dr-hook-timed-kill}}\label{\detokenize{flag/flag:id179}}

\subsection{Valid Values}
\label{\detokenize{flag/flag:id180}}
\begin{DUlineblock}{0em}
\item[] valid\_value ::= \textless{}formatted\textgreater{} |  \textless{}valid\_value\textgreater{} \textless{}delim\textgreater{} \textless{}formatted\textgreater{}
\item[] delim ::= ‘,’ | ‘ ‘ | ‘t’ | ‘/’
\item[] formatted ::= \textless{}id\textgreater{} ‘:’ \textless{}id\textgreater{} ‘:’ \textless{}integer\textgreater{} ‘:’ \textless{}double\textgreater{}
\item[] double ::= \textless{}double\_part\textgreater{} | ‘double\_part’ ‘.’ \textless{}double\_part\textgreater{}
\item[] double\_part ::= \textless{}digit\_0\textgreater{} | \textless{}double\_part\textgreater{} \textless{}digit\_0\textgreater{}
\item[] digit\_0 ::= ‘0’ | ‘1’ | ‘2’ | ‘3’ | ‘4’ | ‘5’ | ‘6’ | ‘7’ | ‘8’ | ‘9’
\item[] id = \textless{}integer\textgreater{} | ‘\sphinxhyphen{}1’
\item[] integer ::= \textless{}digit\textgreater{} | \textless{}integer\textgreater{} \textless{}digit\textgreater{} | \textless{}integer\textgreater{} ‘0’
\item[] digit ::= ‘1’ | ‘2’ | ‘3’ | ‘4’ | ‘5’ | ‘6’ | ‘7’ | ‘8’ | ‘9’
\end{DUlineblock}


\subsection{Purpose}
\label{\detokenize{flag/flag:id181}}
\sphinxAtStartPar
Set timers for specific, or all, threads to trigger signals. This is a developer option.


\subsection{Notes}
\label{\detokenize{flag/flag:id182}}
\sphinxAtStartPar
The parameters provided to \sphinxcode{\sphinxupquote{DR\_HOOK\_TIMED\_KILL}} (\sphinxcode{\sphinxupquote{target\_process}}, \sphinxcode{\sphinxupquote{target\_oml\_thread\_id}}, \sphinxcode{\sphinxupquote{target\_signal}}, and \sphinxcode{\sphinxupquote{start\_time}}) are in the following pattern:

\sphinxAtStartPar
\sphinxcode{\sphinxupquote{target\_process:target\_oml\_thread\_id:target\_signal:start\_time}}

\sphinxAtStartPar
Note that:
\begin{itemize}
\item {} 
\sphinxAtStartPar
\sphinxcode{\sphinxupquote{target\_process}} can be a valid integer for a single process, or \sphinxcode{\sphinxupquote{\sphinxhyphen{}1}} for all.

\item {} 
\sphinxAtStartPar
\sphinxcode{\sphinxupquote{target\_oml\_thread\_id}} can be a valid integer for a single thread, or \sphinxcode{\sphinxupquote{\sphinxhyphen{}1}} for all.

\item {} 
\sphinxAtStartPar
\sphinxcode{\sphinxupquote{target\_signal}} must be valid for your system.

\item {} 
\sphinxAtStartPar
\sphinxcode{\sphinxupquote{start\_time}} must be non\sphinxhyphen{}zero.

\end{itemize}

\sphinxAtStartPar
Any set of parameters that are invalid, will be silently discarded.

\sphinxAtStartPar
The timers will be started as part of drhook’s initialisation, and are specified in seconds.

\sphinxAtStartPar
If this option isn’t specified, then it will default to \sphinxcode{\sphinxupquote{NULL}}.


\section{DR\_HOOK\_DUMP\_SMAPS}
\label{\detokenize{flag/flag:dr-hook-dump-smaps}}\label{\detokenize{flag/flag:id184}}

\subsection{Valid Values}
\label{\detokenize{flag/flag:id185}}
\begin{DUlineblock}{0em}
\item[] valid\_value ::= ‘0’ | ‘1’
\end{DUlineblock}


\subsection{Purpose}
\label{\detokenize{flag/flag:id186}}
\sphinxAtStartPar
Specifies whether \sphinxcode{\sphinxupquote{/proc/\textless{}process\_ID\textgreater{}/smaps}} should be dumped when handling a signal or not. This is a developer option.


\subsection{Notes}
\label{\detokenize{flag/flag:id187}}
\sphinxAtStartPar
\sphinxcode{\sphinxupquote{smaps}} will be dumped after the first signal is handled by drhook, but before signals are handled by other handlers, e.g. \sphinxstyleabbreviation{ATP}. It will only be dumped for the process which raises the signal.

\sphinxAtStartPar
\sphinxcode{\sphinxupquote{smaps}} will be dumped to \sphinxcode{\sphinxupquote{/proc/\textless{}process\_ID\textgreater{}/smaps}}.

\sphinxAtStartPar
Any non\sphinxhyphen{}zero valid integer value will be set to \sphinxcode{\sphinxupquote{1}}.

\sphinxAtStartPar
If this option isn’t specified, then it will default to \sphinxcode{\sphinxupquote{0}}.


\section{DR\_HOOK\_DUMP\_MAPS}
\label{\detokenize{flag/flag:dr-hook-dump-maps}}\label{\detokenize{flag/flag:id190}}

\subsection{Valid Values}
\label{\detokenize{flag/flag:id191}}
\begin{DUlineblock}{0em}
\item[] valid\_value ::= ‘0’ | ‘1’
\end{DUlineblock}


\subsection{Purpose}
\label{\detokenize{flag/flag:id192}}
\sphinxAtStartPar
Specifies whether \sphinxcode{\sphinxupquote{/proc/\textless{}process\_ID\textgreater{}/maps}} should be dumped when handling a signal or not. This is a developer option.


\subsection{Notes}
\label{\detokenize{flag/flag:id193}}
\sphinxAtStartPar
\sphinxcode{\sphinxupquote{maps}} will be dumped after the first signal is handled by drhook, but before signals are handled by other handlers, e.g. \sphinxstyleabbreviation{ATP}. It will only be dumped for the process which raises the signal.

\sphinxAtStartPar
\sphinxcode{\sphinxupquote{maps}} will be dumped to \sphinxcode{\sphinxupquote{/proc/\textless{}process\_ID\textgreater{}/maps}}

\sphinxAtStartPar
Any non\sphinxhyphen{}zero valid integer value will be set to \sphinxcode{\sphinxupquote{1}}.

\sphinxAtStartPar
If this option isn’t specified, then it will default to \sphinxcode{\sphinxupquote{0}}.


\section{DR\_HOOK\_DUMP\_BUDDYINFO}
\label{\detokenize{flag/flag:dr-hook-dump-buddyinfo}}\label{\detokenize{flag/flag:id196}}

\subsection{Valid Values}
\label{\detokenize{flag/flag:id197}}
\begin{DUlineblock}{0em}
\item[] valid\_value ::= ‘0’ | ‘1’
\end{DUlineblock}


\subsection{Purpose}
\label{\detokenize{flag/flag:id198}}
\sphinxAtStartPar
Specifies whether \sphinxcode{\sphinxupquote{/proc/buddyinfo}} should be dumped when handling a signal or not. This is a developer option.


\subsection{Notes}
\label{\detokenize{flag/flag:id199}}
\sphinxAtStartPar
\sphinxcode{\sphinxupquote{buddyinfo}} will be dumped after the first signal is handled by drhook, but before signals are handled by other handlers, e.g. \sphinxstyleabbreviation{ATP}.

\sphinxAtStartPar
\sphinxcode{\sphinxupquote{buddyinfo}} can also be dumped during the dumping of hugepages(enabled via {\hyperref[\detokenize{flag/flag:dr-hook-dump-hugepages}]{\sphinxcrossref{\DUrole{std,std-ref}{DR\_HOOK\_DUMP\_HUGEPAGES}}}}). This occurs during the handling of signals, and when drhook enters a region.

\sphinxAtStartPar
\sphinxcode{\sphinxupquote{buddyinfo}} will be dumped to \sphinxcode{\sphinxupquote{/proc/buddyinfo}}.

\sphinxAtStartPar
Any non\sphinxhyphen{}zero valid integer value will be set to \sphinxcode{\sphinxupquote{1}}.

\sphinxAtStartPar
If this option isn’t specified, then it will default to \sphinxcode{\sphinxupquote{0}}.


\section{DR\_HOOK\_DUMP\_MEMINFO}
\label{\detokenize{flag/flag:dr-hook-dump-meminfo}}\label{\detokenize{flag/flag:id202}}

\subsection{Valid Values}
\label{\detokenize{flag/flag:id203}}
\begin{DUlineblock}{0em}
\item[] valid\_value ::= ‘0’ | ‘1’
\end{DUlineblock}


\subsection{Purpose}
\label{\detokenize{flag/flag:id204}}
\sphinxAtStartPar
Specifies whether \sphinxcode{\sphinxupquote{/proc/meminfo}} should be dumped when handling a signal or not. This is a developer option.


\subsection{Notes}
\label{\detokenize{flag/flag:id205}}
\sphinxAtStartPar
\sphinxcode{\sphinxupquote{meminfo}} will be dumped after the first signal is handled by drhook, but before signals are handled by other handlers, e.g. \sphinxstyleabbreviation{ATP}.

\sphinxAtStartPar
\sphinxcode{\sphinxupquote{meminfo}} can also be dumped during the dumping of hugepages (enabled via {\hyperref[\detokenize{flag/flag:dr-hook-dump-hugepages}]{\sphinxcrossref{\DUrole{std,std-ref}{DR\_HOOK\_DUMP\_HUGEPAGES}}}}). This occurs during the handling of signals, and when drhook enters a region.

\sphinxAtStartPar
\sphinxcode{\sphinxupquote{meminfo}} will be dumped to \sphinxcode{\sphinxupquote{/proc/meminfo}}.

\sphinxAtStartPar
Any non\sphinxhyphen{}zero valid integer value will be set to \sphinxcode{\sphinxupquote{1}}.

\sphinxAtStartPar
If this option isn’t specified, then it will default to \sphinxcode{\sphinxupquote{0}}.


\section{DR\_HOOK\_DUMP\_HUGEPAGES}
\label{\detokenize{flag/flag:dr-hook-dump-hugepages}}\label{\detokenize{flag/flag:id208}}

\subsection{Valid Values}
\label{\detokenize{flag/flag:id209}}
\begin{DUlineblock}{0em}
\item[] valid\_value ::= \textless{}id\textgreater{} ‘,’ \textless{}double\textgreater{}
\item[] double ::= \textless{}double\_part\textgreater{} | ‘double\_part’ ‘.’ \textless{}double\_part\textgreater{}
\item[] double\_part ::= \textless{}digit\_0\textgreater{} | \textless{}double\_part\textgreater{} \textless{}digit\_0\textgreater{}
\item[] digit\_0 ::= ‘0’ | ‘1’ | ‘2’ | ‘3’ | ‘4’ | ‘5’ | ‘6’ | ‘7’ | ‘8’ | ‘9’
\item[] id = \textless{}integer\textgreater{} | ‘\sphinxhyphen{}1’
\item[] integer ::= \textless{}digit\textgreater{} | \textless{}integer\textgreater{} \textless{}digit\textgreater{} | \textless{}integer\textgreater{} ‘0’
\item[] digit ::= ‘1’ | ‘2’ | ‘3’ | ‘4’ | ‘5’ | ‘6’ | ‘7’ | ‘8’ | ‘9’
\end{DUlineblock}


\subsection{Purpose}
\label{\detokenize{flag/flag:id210}}

\subsection{Notes}
\label{\detokenize{flag/flag:id212}}
\sphinxAtStartPar
The parameters provided to \sphinxcode{\sphinxupquote{DR\_HOOK\_DUMP\_HUGEPAGES}} (\sphinxcode{\sphinxupquote{target\_process}}, and \sphinxcode{\sphinxupquote{frequency}}) are in the following pattern:

\sphinxAtStartPar
\sphinxcode{\sphinxupquote{target\_process, frequency}}

\sphinxAtStartPar
Note that:
\begin{itemize}
\item {} 
\sphinxAtStartPar
\sphinxcode{\sphinxupquote{target\_process}} can be a valid integer for a single process, or \sphinxcode{\sphinxupquote{\sphinxhyphen{}1}} for all.

\item {} 
\sphinxAtStartPar
\sphinxcode{\sphinxupquote{frequency}} is given in seconds. This is the minimum time that must pass before huge pages are dumped \sphinxhyphen{} unless the function has been called with the \sphinxcode{\sphinxupquote{enforced}} flag. Although, enforcement isn’t currently used within drhook.

\end{itemize}

\sphinxAtStartPar
If either parameters are invalid, both will be silently discarded.

\sphinxAtStartPar
Hugepages are only dumped by the first thread.

\sphinxAtStartPar
Hugepage dumping can occur when handling a signal or entering a drhook region.

\sphinxAtStartPar
\sphinxcode{\sphinxupquote{/proc/buddyinfo}} and \sphinxcode{\sphinxupquote{/proc/meminfo}} can also be dumped with hugepages, if {\hyperref[\detokenize{flag/flag:dr-hook-dump-buddyinfo}]{\sphinxcrossref{\DUrole{std,std-ref}{DR\_HOOK\_DUMP\_BUDDYINFO}}}} and {\hyperref[\detokenize{flag/flag:dr-hook-dump-meminfo}]{\sphinxcrossref{\DUrole{std,std-ref}{DR\_HOOK\_DUMP\_MEMINFO}}}} are enabled respectively.

\sphinxAtStartPar
The dumping of hugepages is handled by ec\_meminfo.

\sphinxAtStartPar
If this option isn’t specified, then it will default to \sphinxcode{\sphinxupquote{0,0}}.


\section{DR\_HOOK\_GENCORE}
\label{\detokenize{flag/flag:dr-hook-gencore}}\label{\detokenize{flag/flag:id215}}

\subsection{Valid Values}
\label{\detokenize{flag/flag:id216}}
\begin{DUlineblock}{0em}
\item[] valid\_value ::= ‘0’ | ‘1’
\end{DUlineblock}


\subsection{Purpose}
\label{\detokenize{flag/flag:id217}}
\sphinxAtStartPar
Specifies whether generating core dumps is enabled or not.


\subsection{Notes}
\label{\detokenize{flag/flag:id218}}
\sphinxAtStartPar
Any non\sphinxhyphen{}zero valid integer value will be treated as \sphinxcode{\sphinxupquote{1}}.

\sphinxAtStartPar
If this option isn’t specified, then it will default to \sphinxcode{\sphinxupquote{0}}.


\section{DR\_HOOK\_GENCORE\_SIGNAL}
\label{\detokenize{flag/flag:dr-hook-gencore-signal}}\label{\detokenize{flag/flag:id220}}

\subsection{Valid Values}
\label{\detokenize{flag/flag:id221}}
\begin{DUlineblock}{0em}
\item[] valid\_value ::= \textless{}digit\textgreater{} | \textless{}valid\_value\textgreater{} \textless{}digit\textgreater{} | \textless{}valid\_value\textgreater{} ‘0’
\item[] digit ::= ‘1’ | ‘2’ | ‘3’ | ‘4’ | ‘5’ | ‘6’ | ‘7’ | ‘8’ | ‘9’
\end{DUlineblock}


\subsection{Purpose}
\label{\detokenize{flag/flag:id222}}
\sphinxAtStartPar
Specifies the signal which triggers generating a core dump.


\subsection{Notes}
\label{\detokenize{flag/flag:id223}}
\sphinxAtStartPar
To be valid then it must be \(1 \leq\) \sphinxcode{\sphinxupquote{signal}} \(\leq\) \sphinxcode{\sphinxupquote{NSIG}}, and not equal to \sphinxcode{\sphinxupquote{SIGABRT}}. \sphinxcode{\sphinxupquote{NSIG}} and \sphinxcode{\sphinxupquote{SIGABRT}} are defined in \sphinxcode{\sphinxupquote{signal.h}} and are system dependant.

\sphinxAtStartPar
All invalid values will be silently discarded.

\sphinxAtStartPar
If this option isn’t specified, then it will default to \sphinxcode{\sphinxupquote{0}}, which causes it to be ignored.


\section{DR\_HOOK\_STRICT\_REGIONS}
\label{\detokenize{flag/flag:dr-hook-strict-regions}}\label{\detokenize{flag/flag:id225}}

\subsection{Valid Values}
\label{\detokenize{flag/flag:id226}}
\begin{DUlineblock}{0em}
\item[] valid\_value ::= ‘0’ | ‘1’
\end{DUlineblock}


\subsection{Purpose}
\label{\detokenize{flag/flag:id227}}
\sphinxAtStartPar
Specifies if drhook region entry and exit calls must use the same label.


\subsection{Notes}
\label{\detokenize{flag/flag:id228}}
\sphinxAtStartPar
drhook region entry and exit calls can take a string, which is treated as a label for the region. However, while drhook doesn’t use the exit call label itself, extensions that hijack drhook calls may need the label as the unique identifier. This flag will cause drhook to crash instead of passing it to the extension.

\sphinxAtStartPar
Any non\sphinxhyphen{}zero valid integer value will be treated as \sphinxcode{\sphinxupquote{1}}.

\sphinxAtStartPar
If this option isn’t specified and {\hyperref[\detokenize{flag/flag:dr-hook-nvtx}]{\sphinxcrossref{\DUrole{std,std-ref}{DR\_HOOK\_NVTX}}}} is disabled, then it will default to \sphinxcode{\sphinxupquote{0}}. If {\hyperref[\detokenize{flag/flag:dr-hook-nvtx}]{\sphinxcrossref{\DUrole{std,std-ref}{DR\_HOOK\_NVTX}}}} is enabled, then {\hyperref[\detokenize{flag/flag:dr-hook-strict-regions}]{\sphinxcrossref{\DUrole{std,std-ref}{DR\_HOOK\_STRICT\_REGIONS}}}} will be enabled regardless of the value given.


\section{DR\_HOOK\_NVTX}
\label{\detokenize{flag/flag:dr-hook-nvtx}}\label{\detokenize{flag/flag:id229}}

\subsection{Valid Values}
\label{\detokenize{flag/flag:id230}}
\begin{DUlineblock}{0em}
\item[] valid\_value ::= ‘0’ | ‘1’
\end{DUlineblock}


\subsection{Purpose}
\label{\detokenize{flag/flag:id231}}
\sphinxAtStartPar
Specifies if NVTX should be enabled at runtime or not.


\subsection{Notes}
\label{\detokenize{flag/flag:id232}}
\sphinxAtStartPar
Enables Nvidia’s \sphinxhref{https://github.com/NVIDIA/NVTX}{NVTX} at runtime. This assumes that NVTX has been enabled and requested at compile time of drhook using \DUrole{xref,std,std-ref}{ENABLEDR\_HOOK\_NVTX}.

\sphinxAtStartPar
If {\hyperref[\detokenize{flag/flag:dr-hook-nvtx}]{\sphinxcrossref{\DUrole{std,std-ref}{DR\_HOOK\_NVTX}}}} is enabled, then {\hyperref[\detokenize{flag/flag:dr-hook-strict-regions}]{\sphinxcrossref{\DUrole{std,std-ref}{DR\_HOOK\_STRICT\_REGIONS}}}} will also be enabled. It will also enable \sphinxcode{\sphinxupquote{walltime}} and \sphinxcode{\sphinxupquote{count}} from {\hyperref[\detokenize{flag/flag:dr-hook-opt}]{\sphinxcrossref{\DUrole{std,std-ref}{DR\_HOOK\_OPT}}}}.

\sphinxAtStartPar
Setting \sphinxcode{\sphinxupquote{DR\_HOOK\_TIMELINE}} to a non\sphinxhyphen{}zero value also causes drhook to check the following options:
\begin{itemize}
\item {} 
\sphinxAtStartPar
{\hyperref[\detokenize{flag/flag:dr-hook-nvtx-spam-call-count}]{\sphinxcrossref{\DUrole{std,std-ref}{DR\_HOOK\_NVTX\_SPAM\_CALL\_COUNT}}}}

\item {} 
\sphinxAtStartPar
{\hyperref[\detokenize{flag/flag:dr-hook-nvtx-spam-wt}]{\sphinxcrossref{\DUrole{std,std-ref}{DR\_HOOK\_NVTX\_SPAM\_WT}}}}

\end{itemize}

\sphinxAtStartPar
Any non\sphinxhyphen{}zero valid integer value will be treated as \sphinxcode{\sphinxupquote{1}}.

\sphinxAtStartPar
If this option isn’t specified, then it will default to \sphinxcode{\sphinxupquote{0}}.


\section{DR\_HOOK\_NVTX\_SPAM\_CALL\_COUNT}
\label{\detokenize{flag/flag:dr-hook-nvtx-spam-call-count}}\label{\detokenize{flag/flag:id234}}

\subsection{Valid Values}
\label{\detokenize{flag/flag:id235}}
\begin{DUlineblock}{0em}
\item[] valid\_value ::= \textless{}digit\textgreater{} | \textless{}valid\_value\textgreater{} \textless{}digit\textgreater{} | \textless{}valid\_value\textgreater{} ‘0’
\item[] digit ::= ‘1’ | ‘2’ | ‘3’ | ‘4’ | ‘5’ | ‘6’ | ‘7’ | ‘8’ | ‘9’
\end{DUlineblock}


\subsection{Purpose}
\label{\detokenize{flag/flag:id236}}
\sphinxAtStartPar
Specifies the spam call count for NVTX.


\subsection{Notes}
\label{\detokenize{flag/flag:id237}}
\sphinxAtStartPar
Spam call count is the number of times a drhook region has to be called, without a cumulative runtime longer than the {\hyperref[\detokenize{flag/flag:dr-hook-nvtx-spam-wt}]{\sphinxcrossref{\DUrole{std,std-ref}{DR\_HOOK\_NVTX\_SPAM\_WT}}}} time, for drhook to skip subsequent calls of that region for the functionality of NVTX. It will not skip core drhook profiling.

\sphinxAtStartPar
This option is only available if NVTX is enabled via {\hyperref[\detokenize{flag/flag:dr-hook-nvtx}]{\sphinxcrossref{\DUrole{std,std-ref}{DR\_HOOK\_NVTX}}}}.

\sphinxAtStartPar
A value less than \sphinxcode{\sphinxupquote{0}} will be set to the definition of \sphinxcode{\sphinxupquote{nvtx\_SCC\_default}}, typically \sphinxcode{\sphinxupquote{10}}. While it is possible to specify \sphinxcode{\sphinxupquote{0}}, this will effectively disable NVTX as all regions will be skipped \sphinxhyphen{} unless they can accumulate sufficient runtime to satisfy {\hyperref[\detokenize{flag/flag:dr-hook-nvtx-spam-wt}]{\sphinxcrossref{\DUrole{std,std-ref}{DR\_HOOK\_NVTX\_SPAM\_WT}}}} in their first call.

\sphinxAtStartPar
The size is limited to the size of \sphinxcode{\sphinxupquote{%
\PYG{k+kt}{int}%
}}.

\sphinxAtStartPar
If this option isn’t specified, then it will default to the definition of \sphinxcode{\sphinxupquote{nvtx\_SCC\_default}}.


\section{DR\_HOOK\_NVTX\_SPAM\_WT}
\label{\detokenize{flag/flag:dr-hook-nvtx-spam-wt}}\label{\detokenize{flag/flag:id238}}

\subsection{Valid Values}
\label{\detokenize{flag/flag:id239}}
\begin{DUlineblock}{0em}
\item[] valid\_value ::= \textless{}number\textgreater{} | ‘number’ ‘.’ \textless{}number\textgreater{}
\item[] number ::= \textless{}digit\textgreater{} | \textless{}double\_part\textgreater{} \textless{}digit\textgreater{}
\item[] digit ::= ‘0’ | ‘1’ | ‘2’ | ‘3’ | ‘4’ | ‘5’ | ‘6’ | ‘7’ | ‘8’ | ‘9’
\end{DUlineblock}


\subsection{Purpose}
\label{\detokenize{flag/flag:id240}}
\sphinxAtStartPar
Specifies the spam wall time for NVTX.


\subsection{Notes}
\label{\detokenize{flag/flag:id241}}
\sphinxAtStartPar
Spam wall time is the cumulative runtime a drhook region must have if it is not to be skipped for NVTX functionality after exceeding {\hyperref[\detokenize{flag/flag:dr-hook-nvtx-spam-call-count}]{\sphinxcrossref{\DUrole{std,std-ref}{DR\_HOOK\_NVTX\_SPAM\_CALL\_COUNT}}}}. It will not skip core drhook profiling.

\sphinxAtStartPar
This option is only available if NVTX is enabled via {\hyperref[\detokenize{flag/flag:dr-hook-nvtx}]{\sphinxcrossref{\DUrole{std,std-ref}{DR\_HOOK\_NVTX}}}}.

\sphinxAtStartPar
{\hyperref[\detokenize{flag/flag:dr-hook-nvtx-spam-wt}]{\sphinxcrossref{\DUrole{std,std-ref}{DR\_HOOK\_NVTX\_SPAM\_WT}}}} is measured in seconds.

\sphinxAtStartPar
A value less than \sphinxcode{\sphinxupquote{0}} will be set to the definition of \sphinxcode{\sphinxupquote{nvtx\_SWT\_default}}, typically \sphinxcode{\sphinxupquote{0.001}}.

\sphinxAtStartPar
The size is limited to the size of \sphinxcode{\sphinxupquote{%
\PYG{k+kt}{double}%
}}.

\sphinxAtStartPar
If this option isn’t specified, then it will default to the definition of \sphinxcode{\sphinxupquote{nvtx\_SWT\_default}}.


\section{DR\_HOOK\_OPT}
\label{\detokenize{flag/flag:dr-hook-opt}}\label{\detokenize{flag/flag:id242}}

\subsection{Valid Values}
\label{\detokenize{flag/flag:id243}}
\begin{DUlineblock}{0em}
\item[] valid\_value ::= \textless{}option\textgreater{} |  \textless{}valid\_value\textgreater{} \textless{}delim\textgreater{} \textless{}option\textgreater{}
\item[] delim ::= ‘,’ | ‘ ‘ | ‘t’ | ‘/’
\item[] option ::= \textless{}all\textgreater{} | \textless{}memory\textgreater{} | \textless{}heap\textgreater{} | \textless{}stack\textgreater{} | \textless{}rss\textgreater{} | \textless{}paging\textgreater{} | \textless{}walltime\textgreater{} | \textless{}cputime\textgreater{} | \textless{}count\textgreater{} | \textless{}memprof\textgreater{} | \textless{}cycles\textgreater{} | \textless{}cpuprof\textgreater{} | \textless{}trim\textgreater{} | \textless{}self\textgreater{} | \textless{}noself\textgreater{} | \textless{}noprop\textgreater{} | \textless{}nosize\textgreater{} | \textless{}cluster\textgreater{} | \textless{}callpath\textgreater{} | \textless{}papi\textgreater{}
\item[] all ::= ‘ALL’
\item[] memory ::= ‘MEM’ | ‘MEMORY’
\item[] time ::= ‘TIME’ | ‘TIMES’
\item[] heap ::= ‘HWM’ | ‘HEAP’
\item[] stack ::= ‘STK’ | ‘STACK’
\item[] rss ::= ‘RSS’
\item[] paging ::= ‘PAG’ | ‘PAGING’
\item[] walltime ::= ‘WALL’ | ‘WALLTIME’
\item[] cputime ::= ‘CPU’ | ‘CPUTIME’
\item[] count ::= ‘CALLS’ | ‘COUNT’
\item[] memprof ::= ‘MEMPROF’
\item[] cycles ::= ‘PROF’ | ‘WALLPROF’ | ‘CYCLES’
\item[] cpuprof ::= ‘CPUPROF’
\item[] trim ::= ‘TRIM’
\item[] self ::= ‘SELF’
\item[] noself ::= ‘NOSELF’
\item[] noprop ::= ‘NOPROP’ | ‘NOPROPAGATE’ | ‘NOPROPAGATE\_SIGNALS’
\item[] nosize ::= ‘NOSIZE’ | ‘NOSIZEINFO’
\item[] cluster ::= ‘CLUSTER’ | ‘CLUSTERINFO’
\item[] callpath ::= ‘CALLPATH’
\item[] papi ::= ‘COUNTERS’
\end{DUlineblock}


\subsection{Purpose}
\label{\detokenize{flag/flag:id244}}
\sphinxAtStartPar
Specifies a range of options relating to profiling parameters and outputs.


\subsection{Notes}
\label{\detokenize{flag/flag:id245}}

\begin{savenotes}
\sphinxatlongtablestart
\sphinxthistablewithglobalstyle
\makeatletter
  \LTleft \@totalleftmargin plus1fill
  \LTright\dimexpr\columnwidth-\@totalleftmargin-\linewidth\relax plus1fill
\makeatother
\begin{longtable}{\X{4}{37}\X{10}{37}\X{5}{37}\X{6}{37}\X{12}{37}}
\sphinxthelongtablecaptionisattop
\caption{DR\_HOOK\_OPT Options\strut}\label{\detokenize{flag/flag:id285}}\\*[\sphinxlongtablecapskipadjust]
\sphinxtoprule
\sphinxstyletheadfamily 
\sphinxAtStartPar
Flag
&\sphinxstyletheadfamily 
\sphinxAtStartPar
Implements
&\sphinxstyletheadfamily 
\sphinxAtStartPar
Enables
&\sphinxstyletheadfamily 
\sphinxAtStartPar
Increments any\_memstat?
&\sphinxstyletheadfamily 
\sphinxAtStartPar
None
\\
\sphinxmidrule
\endfirsthead

\multicolumn{5}{c}{\sphinxnorowcolor
    \makebox[0pt]{\sphinxtablecontinued{\tablename\ \thetable{} \textendash{} continued from previous page}}%
}\\
\sphinxtoprule
\sphinxstyletheadfamily 
\sphinxAtStartPar
Flag
&\sphinxstyletheadfamily 
\sphinxAtStartPar
Implements
&\sphinxstyletheadfamily 
\sphinxAtStartPar
Enables
&\sphinxstyletheadfamily 
\sphinxAtStartPar
Increments any\_memstat?
&\sphinxstyletheadfamily 
\sphinxAtStartPar
None
\\
\sphinxmidrule
\endhead

\sphinxbottomrule
\multicolumn{5}{r}{\sphinxnorowcolor
    \makebox[0pt][r]{\sphinxtablecontinued{continues on next page}}%
}\\
\endfoot

\endlastfoot
\sphinxtableatstartofbodyhook

\sphinxAtStartPar
all
&
\sphinxAtStartPar
None
&\begin{itemize}
\item {} 
\sphinxAtStartPar
heap

\item {} 
\sphinxAtStartPar
stack

\item {} 
\sphinxAtStartPar
rss

\item {} 
\sphinxAtStartPar
paging

\item {} 
\sphinxAtStartPar
walltime

\item {} 
\sphinxAtStartPar
cputime

\item {} 
\sphinxAtStartPar
cycles

\item {} 
\sphinxAtStartPar
count

\item {} 
\sphinxAtStartPar
papi

\end{itemize}
&
\sphinxAtStartPar
Yes
&
\sphinxAtStartPar
None
\\
\sphinxhline
\sphinxAtStartPar
memory
&
\sphinxAtStartPar
None
&\begin{itemize}
\item {} 
\sphinxAtStartPar
heap

\item {} 
\sphinxAtStartPar
stack

\item {} 
\sphinxAtStartPar
rss

\item {} 
\sphinxAtStartPar
count

\end{itemize}
&
\sphinxAtStartPar
Yes
&
\sphinxAtStartPar
None
\\
\sphinxhline
\sphinxAtStartPar
times
&
\sphinxAtStartPar
None
&\begin{itemize}
\item {} 
\sphinxAtStartPar
walltime

\item {} 
\sphinxAtStartPar
cputime

\item {} 
\sphinxAtStartPar
count

\end{itemize}
&
\sphinxAtStartPar
No
&
\sphinxAtStartPar
None
\\
\sphinxhline
\sphinxAtStartPar
heap
&\begin{itemize}
\item {} 
\sphinxAtStartPar
Track and print current and max heap memory usage.

\end{itemize}
&\begin{itemize}
\item {} 
\sphinxAtStartPar
count

\end{itemize}
&
\sphinxAtStartPar
Yes
&\begin{itemize}
\item {} 
\sphinxAtStartPar
Can also be enabled by {\hyperref[\detokenize{flag/flag:dr-hook-funcenter}]{\sphinxcrossref{\DUrole{std,std-ref}{DR\_HOOK\_FUNCENTER}}}} or {\hyperref[\detokenize{flag/flag:dr-hook-funcexit}]{\sphinxcrossref{\DUrole{std,std-ref}{DR\_HOOK\_FUNCEXIT}}}} being enabled.

\end{itemize}
\\
\sphinxhline
\sphinxAtStartPar
stack
&\begin{itemize}
\item {} 
\sphinxAtStartPar
Track and print current and max stack memory usage.

\end{itemize}
&\begin{itemize}
\item {} 
\sphinxAtStartPar
count

\end{itemize}
&
\sphinxAtStartPar
Yes
&\begin{itemize}
\item {} 
\sphinxAtStartPar
Can also be enabled by {\hyperref[\detokenize{flag/flag:dr-hook-funcenter}]{\sphinxcrossref{\DUrole{std,std-ref}{DR\_HOOK\_FUNCENTER}}}} or {\hyperref[\detokenize{flag/flag:dr-hook-funcexit}]{\sphinxcrossref{\DUrole{std,std-ref}{DR\_HOOK\_FUNCEXIT}}}} being enabled.

\end{itemize}
\\
\sphinxhline
\sphinxAtStartPar
rss
&\begin{itemize}
\item {} 
\sphinxAtStartPar
Track and print current and max resident set size.

\end{itemize}
&\begin{itemize}
\item {} 
\sphinxAtStartPar
count

\end{itemize}
&
\sphinxAtStartPar
Yes
&
\sphinxAtStartPar
None
\\
\sphinxhline
\sphinxAtStartPar
paging
&\begin{itemize}
\item {} 
\sphinxAtStartPar
Track and print current number of allocated pages.

\item {} 
\sphinxAtStartPar
Also tracks the maximum number of pages allocated per drhook region.

\end{itemize}
&\begin{itemize}
\item {} 
\sphinxAtStartPar
count

\end{itemize}
&
\sphinxAtStartPar
Yes
&
\sphinxAtStartPar
None
\\
\sphinxhline
\sphinxAtStartPar
walltime
&\begin{itemize}
\item {} 
\sphinxAtStartPar
Tracks total and per drhook region walltime.

\end{itemize}
&\begin{itemize}
\item {} 
\sphinxAtStartPar
count

\end{itemize}
&
\sphinxAtStartPar
No
&\begin{itemize}
\item {} 
\sphinxAtStartPar
\sphinxcode{\sphinxupquote{walltime}} is measured in seconds.

\end{itemize}
\\
\sphinxhline
\sphinxAtStartPar
cputime
&\begin{itemize}
\item {} 
\sphinxAtStartPar
Tracks total and per drhook region cputime per process.

\end{itemize}
&\begin{itemize}
\item {} 
\sphinxAtStartPar
count

\end{itemize}
&
\sphinxAtStartPar
No
&\begin{itemize}
\item {} 
\sphinxAtStartPar
\sphinxcode{\sphinxupquote{cputime}} is measured in CPU ticks.

\item {} 
\sphinxAtStartPar
Terminated child processes’ time will also be included in this figure, i.e., multiprocessing will be reflected in the count.

\end{itemize}
\\
\sphinxhline
\sphinxAtStartPar
count
&\begin{itemize}
\item {} 
\sphinxAtStartPar
Tracks the total number of drhook regions entered (\sphinxcode{\sphinxupquote{calls}}).

\item {} 
\sphinxAtStartPar
Also tracks how deep drhook is in nested regions (\sphinxcode{\sphinxupquote{status}}).

\end{itemize}
&
\sphinxAtStartPar
None
&
\sphinxAtStartPar
No
&\begin{itemize}
\item {} 
\sphinxAtStartPar
drhook uses \sphinxcode{\sphinxupquote{status}} purely for tracking if it is currently in a region, and does not care about the depth.

\end{itemize}
\\
\sphinxhline
\sphinxAtStartPar
memprof
&\begin{itemize}
\item {} 
\sphinxAtStartPar
Unsure

\end{itemize}
&\begin{itemize}
\item {} 
\sphinxAtStartPar
heap

\item {} 
\sphinxAtStartPar
stack

\item {} 
\sphinxAtStartPar
rss

\item {} 
\sphinxAtStartPar
paging

\item {} 
\sphinxAtStartPar
count

\end{itemize}
&
\sphinxAtStartPar
Yes
&
\sphinxAtStartPar
None
\\
\sphinxhline
\sphinxAtStartPar
cycles
&\begin{itemize}
\item {} 
\sphinxAtStartPar
Tracks CPU cycles (\sphinxcode{\sphinxupquote{cycles}}).

\item {} 
\sphinxAtStartPar
Also tracks walltime (\sphinxcode{\sphinxupquote{wallprof}}) for each drhook region.

\end{itemize}
&\begin{itemize}
\item {} 
\sphinxAtStartPar
walltime

\item {} 
\sphinxAtStartPar
count

\end{itemize}
&
\sphinxAtStartPar
No
&\begin{itemize}
\item {} 
\sphinxAtStartPar
Adds printing functions to \sphinxcode{\sphinxupquote{atexit()}}.

\item {} 
\sphinxAtStartPar
Disables cpuprof.

\end{itemize}
\\
\sphinxhline
\sphinxAtStartPar
cpuprof
&\begin{itemize}
\item {} 
\sphinxAtStartPar
Unsure

\end{itemize}
&\begin{itemize}
\item {} 
\sphinxAtStartPar
cputime

\item {} 
\sphinxAtStartPar
count

\end{itemize}
&
\sphinxAtStartPar
No
&\begin{itemize}
\item {} 
\sphinxAtStartPar
Disables cycles.

\item {} 
\sphinxAtStartPar
Also adds printing functions to \sphinxcode{\sphinxupquote{atexit()}}.

\end{itemize}
\\
\sphinxhline
\sphinxAtStartPar
trim
&\begin{itemize}
\item {} 
\sphinxAtStartPar
Trims drhook region names to remove leading spaces and characters after the first subsequent space.
\begin{itemize}
\item {} 
\sphinxAtStartPar
e.g. “  Hello World!” becomes “Hello”.

\end{itemize}

\item {} 
\sphinxAtStartPar
Also converts drhook region names to upper case.

\end{itemize}
&
\sphinxAtStartPar
None
&
\sphinxAtStartPar
No
&
\sphinxAtStartPar
None
\\
\sphinxhline
\sphinxAtStartPar
self
&\begin{itemize}
\item {} 
\sphinxAtStartPar
Includes drhook in profiling, also prints it.

\end{itemize}
&
\sphinxAtStartPar
None
&
\sphinxAtStartPar
No
&\begin{itemize}
\item {} 
\sphinxAtStartPar
Very expensive

\item {} 
\sphinxAtStartPar
The default is to include drhook in profiling, but doesn’t print it.

\end{itemize}
\\
\sphinxhline
\sphinxAtStartPar
noself
&\begin{itemize}
\item {} 
\sphinxAtStartPar
Excludes drhook from profiling.

\end{itemize}
&
\sphinxAtStartPar
None
&
\sphinxAtStartPar
No
&\begin{itemize}
\item {} 
\sphinxAtStartPar
The default is to include drhook in profiling, but doesn’t print it.

\end{itemize}
\\
\sphinxhline
\sphinxAtStartPar
noprop
&\begin{itemize}
\item {} 
\sphinxAtStartPar
Does not propagate handled signals beyond drhook.

\end{itemize}
&
\sphinxAtStartPar
None
&
\sphinxAtStartPar
No
&
\sphinxAtStartPar
None
\\
\sphinxhline
\sphinxAtStartPar
nosize
&\begin{itemize}
\item {} 
\sphinxAtStartPar
Unsure

\end{itemize}
&
\sphinxAtStartPar
None
&
\sphinxAtStartPar
No
&
\sphinxAtStartPar
None
\\
\sphinxhline
\sphinxAtStartPar
cluster
&\begin{itemize}
\item {} 
\sphinxAtStartPar
Prints cluster ID and size.

\end{itemize}
&
\sphinxAtStartPar
None
&
\sphinxAtStartPar
No
&
\sphinxAtStartPar
None
\\
\sphinxhline
\sphinxAtStartPar
callpath
&\begin{itemize}
\item {} 
\sphinxAtStartPar
Tracks and outputs the callpath and depth of drhook regions.

\end{itemize}
&
\sphinxAtStartPar
None
&
\sphinxAtStartPar
No
&\begin{itemize}
\item {} 
\sphinxAtStartPar
Creates \sphinxstyleemphasis{much} more overhead.

\item {} 
\sphinxAtStartPar
Enables callpath mode and associated arguments.

\item {} 
\sphinxAtStartPar
By default, this is to a max depth of 50, but can be changed by {\hyperref[\detokenize{flag/flag:dr-hook-callpath-depth}]{\sphinxcrossref{\DUrole{std,std-ref}{DR\_HOOK\_CALLPATH\_DEPTH}}}}.

\end{itemize}
\\
\sphinxhline
\sphinxAtStartPar
papi
&\begin{itemize}
\item {} 
\sphinxAtStartPar
Enables PAPI mode

\end{itemize}
&\begin{itemize}
\item {} 
\sphinxAtStartPar
cycles

\item {} 
\sphinxAtStartPar
walltime

\item {} 
\sphinxAtStartPar
calls

\item {} 
\sphinxAtStartPar
cycles

\end{itemize}
&
\sphinxAtStartPar
No
&\begin{itemize}
\item {} 
\sphinxAtStartPar
Start and stops PAPI instrumentation with drhook regions

\item {} 
\sphinxAtStartPar
Counters being monitored by PAPI can be set with {\hyperref[\detokenize{flag/flag:dr-hook-papi-counters}]{\sphinxcrossref{\DUrole{std,std-ref}{DR\_HOOK\_PAPI\_COUNTERS}}}}

\item {} 
\sphinxAtStartPar
Disables cpuprof

\end{itemize}
\\
\sphinxbottomrule
\end{longtable}
\sphinxtableafterendhook
\sphinxatlongtableend
\end{savenotes}


\section{DR\_HOOK\_CALLPATH\_INDENT}
\label{\detokenize{flag/flag:dr-hook-callpath-indent}}\label{\detokenize{flag/flag:id254}}

\subsection{Valid Values}
\label{\detokenize{flag/flag:id255}}
\begin{DUlineblock}{0em}
\item[] valid\_value ::= ‘1’ | ‘2’ | ‘3’ | ‘4’ | ‘5’ | ‘6’ | ‘7’ | ‘8’
\end{DUlineblock}


\subsection{Purpose}
\label{\detokenize{flag/flag:id256}}
\sphinxAtStartPar
Specifies the number of spaces to indent callpath output by.


\subsection{Notes}
\label{\detokenize{flag/flag:id257}}
\sphinxAtStartPar
This will only be enabled if {\hyperref[\detokenize{flag/flag:dr-hook-opt}]{\sphinxcrossref{\DUrole{std,std-ref}{DR\_HOOK\_OPT}}}} has \sphinxcode{\sphinxupquote{CALLPATH}} set.

\sphinxAtStartPar
If \sphinxcode{\sphinxupquote{DR\_HOOK\_CALLPATH\_INDENT}} \(< 1\) or \sphinxcode{\sphinxupquote{DR\_HOOK\_CALLPATH\_INDENT}} \(> 8\), then it will be set to \sphinxcode{\sphinxupquote{callpath\_indent\_default}}, which evaluates to \sphinxcode{\sphinxupquote{2}}.

\sphinxAtStartPar
If this option isn’t specified, then it will default to \sphinxcode{\sphinxupquote{callpath\_indent\_default}}.


\section{DR\_HOOK\_CALLPATH\_DEPTH}
\label{\detokenize{flag/flag:dr-hook-callpath-depth}}\label{\detokenize{flag/flag:id259}}

\subsection{Valid Values}
\label{\detokenize{flag/flag:id260}}
\begin{DUlineblock}{0em}
\item[] valid\_value ::= \textless{}digit\textgreater{} | \textless{}valid\_value\textgreater{} \textless{}digit\textgreater{}
\item[] digit ::= ‘0’ | ‘1’ | ‘2’ | ‘3’ | ‘4’ | ‘5’ | ‘6’ | ‘7’ | ‘8’ | ‘9’
\end{DUlineblock}


\subsection{Purpose}
\label{\detokenize{flag/flag:id261}}
\sphinxAtStartPar
Set the maximum depth for tracking callpaths of nested drhook regions.


\subsection{Notes}
\label{\detokenize{flag/flag:id262}}
\sphinxAtStartPar
This will only be enabled if {\hyperref[\detokenize{flag/flag:dr-hook-opt}]{\sphinxcrossref{\DUrole{std,std-ref}{DR\_HOOK\_OPT}}}} has \sphinxcode{\sphinxupquote{CALLPATH}} set.

\sphinxAtStartPar
Caution should be used when picking this value, as being too high will greatly increase both the processing and memory overhead of drhook.

\sphinxAtStartPar
The size is limited to the size of \sphinxcode{\sphinxupquote{%
\PYG{k+kt}{int}%
}}.

\sphinxAtStartPar
If \sphinxcode{\sphinxupquote{DR\_HOOK\_CALLPATH\_DEPTH}} is less than 0, then it will be set to \sphinxcode{\sphinxupquote{callpath\_depth\_default}}, which evaluates to \sphinxcode{\sphinxupquote{50}}. A size of zero will only print the first drhook region in the callpath.

\sphinxAtStartPar
If this option isn’t specified, then it will default to \sphinxcode{\sphinxupquote{callpath\_depth\_default}}.


\section{DR\_HOOK\_CALLPATH\_PACKED}
\label{\detokenize{flag/flag:dr-hook-callpath-packed}}\label{\detokenize{flag/flag:id264}}

\subsection{Valid Values}
\label{\detokenize{flag/flag:id265}}
\begin{DUlineblock}{0em}
\item[] valid\_value ::= ‘0’ | ‘1’
\end{DUlineblock}


\subsection{Purpose}
\label{\detokenize{flag/flag:id266}}
\sphinxAtStartPar
Outputs callpath information in a more compact format.


\subsection{Notes}
\label{\detokenize{flag/flag:id267}}
\sphinxAtStartPar
This will only be enabled if {\hyperref[\detokenize{flag/flag:dr-hook-opt}]{\sphinxcrossref{\DUrole{std,std-ref}{DR\_HOOK\_OPT}}}} has \sphinxcode{\sphinxupquote{CALLPATH}} set.

\sphinxAtStartPar
Any non\sphinxhyphen{}zero valid integer value will be set to \sphinxcode{\sphinxupquote{1}}.

\sphinxAtStartPar
If this option isn’t specified, then it will default to \sphinxcode{\sphinxupquote{0}}.


\section{DR\_HOOK\_CALLTRACE}
\label{\detokenize{flag/flag:dr-hook-calltrace}}\label{\detokenize{flag/flag:id270}}

\subsection{Valid Values}
\label{\detokenize{flag/flag:id271}}
\begin{DUlineblock}{0em}
\item[] valid\_value ::= ‘0’ | ‘1’
\end{DUlineblock}


\subsection{Purpose}
\label{\detokenize{flag/flag:id272}}
\sphinxAtStartPar
Specifies if calltrace mode should be enabled or not.


\subsection{Notes}
\label{\detokenize{flag/flag:id273}}
\sphinxAtStartPar
This will only be enabled if {\hyperref[\detokenize{flag/flag:dr-hook-opt}]{\sphinxcrossref{\DUrole{std,std-ref}{DR\_HOOK\_OPT}}}} has \sphinxcode{\sphinxupquote{CALLPATH}} set.

\sphinxAtStartPar
Any non\sphinxhyphen{}zero valid integer value will be set to \sphinxcode{\sphinxupquote{1}}.

\sphinxAtStartPar
If this option isn’t specified, then it will default to \sphinxcode{\sphinxupquote{0}}.


\section{DR\_HOOK\_WATCH\_PRINT\_MAX}
\label{\detokenize{flag/flag:dr-hook-watch-print-max}}\label{\detokenize{flag/flag:id276}}

\subsection{Valid Values}
\label{\detokenize{flag/flag:id277}}
\begin{DUlineblock}{0em}
\item[] valid\_value ::= \textless{}digit\textgreater{} | \textless{}valid\_value\textgreater{} \textless{}digit\textgreater{}
\item[] digit ::= ‘0’ | ‘1’ | ‘2’ | ‘3’ | ‘4’ | ‘5’ | ‘6’ | ‘7’ | ‘8’ | ‘9’
\end{DUlineblock}


\subsection{Purpose}
\label{\detokenize{flag/flag:id278}}
\sphinxAtStartPar
Specifies the max number of elements per array to be printed per watchpoint.


\subsection{Notes}
\label{\detokenize{flag/flag:id279}}
\sphinxAtStartPar
If \sphinxcode{\sphinxupquote{DR\_HOOK\_WATCH\_PRINT\_MAX}} \(\geq 0\), and less than the number of elements, then the number of elements up to the value of \sphinxcode{\sphinxupquote{DR\_HOOK\_WATCH\_PRINT\_MAX}} will be printed.

\sphinxAtStartPar
The size is limited to the size of \sphinxcode{\sphinxupquote{%
\PYG{k+kt}{int}%
}}.

\sphinxAtStartPar
If this option isn’t specified, then it will default to \sphinxcode{\sphinxupquote{\sphinxhyphen{}1}}.


\section{DR\_HOOK\_PAPI\_COUNTERS}
\label{\detokenize{flag/flag:dr-hook-papi-counters}}\label{\detokenize{flag/flag:id281}}

\subsection{Valid Values}
\label{\detokenize{flag/flag:id282}}
\begin{DUlineblock}{0em}
\item[] valid\_value ::= \textless{}counter\textgreater{} | \textless{}valid\_value\textgreater{} \textless{}delim\textgreater{} \textless{}counter\textgreater{}
\item[] delim ::= ‘,’ | ‘ ‘ | ‘t’ | ‘/’
\item[] counter ::= \textless{}char\textgreater{} | \textless{}char\textgreater{} \textless{}char\textgreater{}
\item[] char ::= \textless{}letter\textgreater{} | \textless{}digit\textgreater{} | \textless{}symbol\textgreater{}
\item[] letter ::= \textless{}lower\textgreater{} | \textless{}upper\textgreater{}
\item[] upper ::= ‘A’ | ‘B’ | ‘C’ | ‘D’ | ‘E’ | ‘F’ | ‘G’ | ‘H’ | ‘I’ | ‘J’ | ‘K’ | ‘L’ | ‘M’ | ‘N’ | ‘O’ | ‘P’ | ‘Q’ | ‘R’ | ‘S’ | ‘T’ | ‘U’ | ‘V’ | ‘W’ | ‘X’ | ‘Y’ | ‘Z’
\item[] lower ::= ‘a’ | ‘b’ | ‘c’ | ‘d’ | ‘e’ | ‘f’ | ‘g’ | ‘h’ | ‘i’ | ‘j’ | ‘k’ | ‘l’ | ‘m’ | ‘n’ | ‘o’ | ‘p’ | ‘q’ | ‘r’ | ‘s’ | ‘t’ | ‘u’ | ‘v’ | ‘w’ | ‘x’ | ‘y’ | ‘z’
\item[] digit ::= ‘0’ | ‘1’ | ‘2’ | ‘3’ | ‘4’ | ‘5’ | ‘6’ | ‘7’ | ‘8’ | ‘9’
\item[] symbol ::= ‘\_’
\end{DUlineblock}


\subsection{Purpose}
\label{\detokenize{flag/flag:id283}}
\sphinxAtStartPar
Specifies the counters to be monitored when in PAPI mode.


\subsection{Notes}
\label{\detokenize{flag/flag:id284}}
\sphinxAtStartPar
\sphinxhref{https://icl.utk.edu/papi/}{PAPI} counters are machine specific and can displayed with the \sphinxstyleliteralstrong{\sphinxupquote{papi avail}} command.

\sphinxAtStartPar
PAPI mode is enabled through {\hyperref[\detokenize{flag/flag:dr-hook-opt}]{\sphinxcrossref{\DUrole{std,std-ref}{DR\_HOOK\_OPT}}}}.

\sphinxAtStartPar
The output from PAPI mode is in csv format and the file path is described, and changed, by {\hyperref[\detokenize{flag/flag:dr-hook-profile}]{\sphinxcrossref{\DUrole{std,std-ref}{DR\_HOOK\_PROFILE}}}}.

\sphinxAtStartPar
A maximum of 4 counters can be specified. Any counters over this limit will be silently dropped.

\sphinxAtStartPar
If this option isn’t specified, then it will default to the following 4 counters:

\begin{DUlineblock}{0em}
\item[] \sphinxcode{\sphinxupquote{API\_TOT\_CYC}}
\item[] \sphinxcode{\sphinxupquote{PAPI\_FP\_OPS}}
\item[] \sphinxcode{\sphinxupquote{PAPI\_L1\_DCA}}
\item[] \sphinxcode{\sphinxupquote{PAPI\_L2\_DCM}}
\end{DUlineblock}

\sphinxstepscope


\chapter{Definitions}
\label{\detokenize{definitions/definitions:definitions}}\label{\detokenize{definitions/definitions::doc}}

\section{\_DRHOOK\_C\_}
\label{\detokenize{definitions/definitions:drhook-c}}\label{\detokenize{definitions/definitions:id1}}

\subsection{Value}
\label{\detokenize{definitions/definitions:value}}
\sphinxAtStartPar
\sphinxcode{\sphinxupquote{%
\PYG{l+m+mi}{1}%
}}


\subsection{Purpose}
\label{\detokenize{definitions/definitions:purpose}}
\sphinxAtStartPar
Never used within \sphinxcode{\sphinxupquote{drhook.c}}


\subsection{Preprocessor Guards}
\label{\detokenize{definitions/definitions:preprocessor-guards}}
\sphinxAtStartPar
None


\section{\_DRHOOK\_FILE\_}
\label{\detokenize{definitions/definitions:drhook-file}}\label{\detokenize{definitions/definitions:id3}}

\subsection{Value}
\label{\detokenize{definitions/definitions:id4}}
\sphinxAtStartPar
\sphinxcode{\sphinxupquote{drhook.c}}


\subsection{Purpose}
\label{\detokenize{definitions/definitions:id5}}
\sphinxAtStartPar
Used within error handling to specify the source file where it occurred.


\subsection{Preprocessor Guards}
\label{\detokenize{definitions/definitions:id6}}
\sphinxAtStartPar
None


\section{\_GNU\_SOURCE}
\label{\detokenize{definitions/definitions:gnu-source}}\label{\detokenize{definitions/definitions:id7}}

\subsection{Value}
\label{\detokenize{definitions/definitions:id8}}
\sphinxAtStartPar
No value defined


\subsection{Purpose}
\label{\detokenize{definitions/definitions:id9}}
\sphinxAtStartPar
Never used within \sphinxcode{\sphinxupquote{drhook.c}}.


\subsection{Preprocessor Guards}
\label{\detokenize{definitions/definitions:id11}}
\sphinxAtStartPar
None


\section{HOST\_NAME\_MAX}
\label{\detokenize{definitions/definitions:host-name-max}}\label{\detokenize{definitions/definitions:id12}}

\subsection{Value}
\label{\detokenize{definitions/definitions:id13}}
\sphinxAtStartPar
\sphinxcode{\sphinxupquote{%
\PYG{n}{\PYGZus{}POSIX\PYGZus{}HOST\PYGZus{}NAME\PYGZus{}MAX}%
}}


\subsection{Purpose}
\label{\detokenize{definitions/definitions:id14}}
\sphinxAtStartPar
Used to set the max size of char arrays for node HOSTNAME.


\subsection{Preprocessor Guards}
\label{\detokenize{definitions/definitions:id15}}
\sphinxAtStartPar
\sphinxcode{\sphinxupquote{%
\PYG{o}{!}\PYG{n}{defined}\PYG{p}{(}\PYG{n}{HOST\PYGZus{}NAME\PYGZus{}MAX}\PYG{p}{)}\PYG{+w}{ }\PYG{o}{\PYGZam{}}\PYG{o}{\PYGZam{}}\PYG{+w}{ }\PYG{n}{defined}\PYG{p}{(}\PYG{n}{\PYGZus{}POSIX\PYGZus{}HOST\PYGZus{}NAME\PYGZus{}MAX}\PYG{p}{)}%
}}


\section{HOST\_NAME\_MAX}
\label{\detokenize{definitions/definitions:id16}}\label{\detokenize{definitions/definitions:id17}}

\subsection{Value}
\label{\detokenize{definitions/definitions:id18}}
\sphinxAtStartPar
\sphinxcode{\sphinxupquote{%
\PYG{n}{\PYGZus{}SC\PYGZus{}HOST\PYGZus{}NAME\PYGZus{}MAX}%
}}


\subsection{Purpose}
\label{\detokenize{definitions/definitions:id19}}
\sphinxAtStartPar
Used to set the max size of char arrays for node HOSTNAME.


\subsection{Preprocessor Guards}
\label{\detokenize{definitions/definitions:id20}}
\sphinxAtStartPar
\sphinxcode{\sphinxupquote{%
\PYG{o}{!}\PYG{n}{defined}\PYG{p}{(}\PYG{n}{HOST\PYGZus{}NAME\PYGZus{}MAX}\PYG{p}{)}\PYG{+w}{ }\PYG{o}{\PYGZam{}}\PYG{o}{\PYGZam{}}\PYG{+w}{ }\PYG{n}{defined}\PYG{p}{(}\PYG{n}{\PYGZus{}SC\PYGZus{}HOST\PYGZus{}NAME\PYGZus{}MAX}\PYG{p}{)}%
}}


\section{EC\_HOST\_NAME\_MAX}
\label{\detokenize{definitions/definitions:ec-host-name-max}}\label{\detokenize{definitions/definitions:id21}}

\subsection{Value}
\label{\detokenize{definitions/definitions:id22}}
\sphinxAtStartPar
\sphinxcode{\sphinxupquote{%
\PYG{n}{HOST\PYGZus{}NAME\PYGZus{}MAX}%
}}


\subsection{Purpose}
\label{\detokenize{definitions/definitions:id23}}
\sphinxAtStartPar
Used to set the max size of char arrays for node HOSTNAME.


\subsection{Preprocessor Guards}
\label{\detokenize{definitions/definitions:id24}}
\sphinxAtStartPar
\sphinxcode{\sphinxupquote{%
\PYG{n}{defined}\PYG{p}{(}\PYG{n}{HOST\PYGZus{}NAME\PYGZus{}MAX}\PYG{p}{)}%
}}


\section{EC\_HOST\_NAME\_MAX}
\label{\detokenize{definitions/definitions:id25}}\label{\detokenize{definitions/definitions:id26}}

\subsection{Value}
\label{\detokenize{definitions/definitions:id27}}
\sphinxAtStartPar
\sphinxcode{\sphinxupquote{%
\PYG{l+m+mi}{512}%
}}


\subsection{Purpose}
\label{\detokenize{definitions/definitions:id28}}
\sphinxAtStartPar
Used to set the max size of char arrays for node HOSTNAME.


\subsection{Preprocessor Guards}
\label{\detokenize{definitions/definitions:id29}}
\sphinxAtStartPar
\sphinxcode{\sphinxupquote{%
\PYG{o}{!}\PYG{n}{defined}\PYG{p}{(}\PYG{n}{HOST\PYGZus{}NAME\PYGZus{}MAX}\PYG{p}{)}%
}}

\sphinxstepscope


\chapter{Global Variables}
\label{\detokenize{global_vars/global_vars:global-variables}}\label{\detokenize{global_vars/global_vars::doc}}

\section{thread\_cycles}
\label{\detokenize{global_vars/global_vars:thread-cycles}}\label{\detokenize{global_vars/global_vars:id1}}

\subsection{Type}
\label{\detokenize{global_vars/global_vars:type}}
\sphinxAtStartPar
\sphinxcode{\sphinxupquote{%
\PYG{k}{static}\PYG{+w}{ }\PYG{k+kt}{long}\PYG{+w}{ }\PYG{k+kt}{long}\PYG{+w}{ }\PYG{k+kt}{int}\PYG{o}{*}%
}}


\subsection{Value}
\label{\detokenize{global_vars/global_vars:value}}
\sphinxAtStartPar
\sphinxcode{\sphinxupquote{%
\PYG{n+nb}{NULL}%
}}


\subsection{Purpose}
\label{\detokenize{global_vars/global_vars:purpose}}
\sphinxAtStartPar
Stores the number of cycles for each thread.


\section{drhook\_lhook}
\label{\detokenize{global_vars/global_vars:drhook-lhook}}\label{\detokenize{global_vars/global_vars:id2}}

\subsection{Type}
\label{\detokenize{global_vars/global_vars:id3}}
\sphinxAtStartPar
\sphinxcode{\sphinxupquote{%
\PYG{k+kt}{int}%
}}


\subsection{Value}
\label{\detokenize{global_vars/global_vars:id4}}
\sphinxAtStartPar
\sphinxcode{\sphinxupquote{%
\PYG{l+m+mi}{1}%
}}


\subsection{Purpose}
\label{\detokenize{global_vars/global_vars:id5}}
\sphinxstepscope


\chapter{Functions}
\label{\detokenize{functions/functions:functions}}\label{\detokenize{functions/functions::doc}}

\section{backtrace()}
\label{\detokenize{functions/functions:backtrace}}\label{\detokenize{functions/functions:id1}}

\subsection{Signature}
\label{\detokenize{functions/functions:signature}}
\sphinxAtStartPar
\sphinxcode{\sphinxupquote{%
\PYG{k}{static}\PYG{+w}{ }\PYG{k+kt}{int}\PYG{+w}{ }\PYG{n}{backtrace}\PYG{p}{(}\PYG{k+kt}{void}\PYG{+w}{ }\PYG{o}{*}\PYG{o}{*}\PYG{n}{buffer}\PYG{p}{,}\PYG{+w}{ }\PYG{k+kt}{int}\PYG{+w}{ }\PYG{n}{size}\PYG{p}{)}%
}}


\subsection{Purpose}
\label{\detokenize{functions/functions:purpose}}
\sphinxAtStartPar
Assumption: Used to provide a dummy function implementation, which is commonly defined by compilers other than the \sphinxstyleemphasis{NEC} compiler


\subsection{Notes}
\label{\detokenize{functions/functions:notes}}
\sphinxAtStartPar
Behind preprocessor guard \sphinxcode{\sphinxupquote{%
\PYG{c+cp}{\PYGZsh{}}\PYG{c+cp}{ifdef\PYGZus{}\PYGZus{}NEC\PYGZus{}\PYGZus{}}%
}}



\renewcommand{\indexname}{Index}
\printindex
\end{document}